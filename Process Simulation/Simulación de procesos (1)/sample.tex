%----------------------------------------------------------------------------------------
%	PACKAGES AND THEMES
%----------------------------------------------------------------------------------------

\documentclass[aspectratio=169,xcolor=dvipsnames]{beamer}
\usetheme{SimpleDarkBlue}

\usepackage[spanish]{babel}
\usepackage{hyperref}
\usepackage{graphicx} % Allows including images
\usepackage{booktabs} % Allows the use of \toprule, \midrule and \bottomrule in tables
\usepackage{amsmath}
\usepackage{lettrine}
\setbeamertemplate{caption}[numbered]
\usepackage[dvipsnames,svgnames,x11names]{xcolor}% Para definir y usar colores (ej. en hipervínculos)
\usepackage{xurl}
\usepackage{hyperref}       % Para crear hipervínculos internos y externos
\hypersetup{
    colorlinks=true,        % Colorear los enlaces en lugar de usar recuadros
    linkcolor=blue,     % Color para enlaces internos (índice, referencias cruzadas)
    filecolor=blue, % Color para enlaces a archivos locales
    urlcolor=blue,      % Color para URLs
    citecolor=blue,     % Color para citas bibliográficas
}
%----------------------------------------------------------------------------------------
%	TITLE PAGE
%----------------------------------------------------------------------------------------

\title{Introducción al Modelado y Simulación de Sistemas}
\subtitle{Materia: Simulación de procesos}

\author{Prof. D.Sc. BARSEKH-ONJI Aboud}

\institute
{
    Facultad de Ingeniería \\
    Universidad Anáhuac México % Your institution for the title page
}
\date{\today} % Date, can be changed to a custom date

%----------------------------------------------------------------------------------------
%	PRESENTATION SLIDES
%----------------------------------------------------------------------------------------
% Poner esto en el preámbulo
\AtBeginSection[]
{
  \begin{frame}{Índice}
    \tableofcontents[currentsection]
  \end{frame}
}
\begin{document}

\begin{frame}
    % Print the title page as the first slide
    \titlepage
\end{frame}


%------------------------------------------------
\section{Introducción a la Teoría de Sistemas}
%------------------------------------------------

\begin{frame}{¿Qué es un Sistema?}

    \begin{block}{Definición Rigurosa}
        Un \textbf{sistema} es una colección de entidades (o componentes) que actúan e interactúan entre sí para lograr un objetivo lógico común.
    \end{block}
    \begin{itemize}
        \item Un sistema está definido por un \textbf{límite} que lo separa de su \textbf{entorno}.
        \item La elección del límite es una decisión crítica en el modelado.
        \item \textbf{Endógeno:} Elementos dentro del sistema.
        \item \textbf{Exógeno:} Elementos fuera del sistema (en el entorno).
    \end{itemize}
\end{frame}

%------------------------------------------------

\begin{frame}{Características de un Sistema}
    \begin{columns}[c]
        \column{.5\textwidth}
            Dentro de sus límites, un sistema se caracteriza por:
            \begin{itemize}
                \item \textbf{Componentes o Entidades:} Elementos discretos (e.g., camiones, operarios, paquetes).
                \item \textbf{Atributos:} Propiedades de las entidades (e.g., capacidad, velocidad, estado).
                \item \textbf{Actividades:} Procesos que consumen tiempo y causan cambios (e.g., carga de un camión, viaje).
                \item \textbf{Estado del Sistema:} Descripción completa del sistema en un punto específico del tiempo.
            \end{itemize}

        \column{.5\textwidth}
            \begin{figure}
                \includegraphics[width=\textwidth]{Figuras/Cap1/fig1.1.png}
                \caption{Representación conceptual de un sistema y su interacción con el entorno.}
                \label{fig:1.1}
            \end{figure}
    \end{columns}
\end{frame}

%------------------------------------------------

\begin{frame}{Ejemplo: Una Fábrica como Sistema}
    \begin{block}{Aplicando los Conceptos}
    Consideremos una fábrica como sistema:
    \end{block}
        \begin{itemize}
            \item \textbf{Componentes:} Trabajadores, máquinas, materias primas.
            \item \textbf{Atributos:} Velocidad de producción, tasa de fallos de las máquinas.
            \item \textbf{Actividades:} Fabricación de piezas, ensamblaje, mantenimiento.
            \item \textbf{Estado:} Número de piezas en buffer, estado de cada máquina.
            \item \textbf{Entorno:} Proveedores, clientes, factores económicos.
        \end{itemize}
\end{frame}

%------------------------------------------------

\begin{frame}{Límites del Sistema}
    \frametitle{Delimitando el Campo de Estudio}
    \begin{alertblock}{Definición del Límite}
    Es una partición que diferencia las entidades del entorno. La elección del límite depende del observador, del tiempo y de la naturaleza del estudio.
    \end{alertblock}
    \begin{itemize}
        \item Puede ser material (piel de un cuerpo) o inmaterial (pertenencia a un grupo).
        \item Determina qué puede entrar o salir del sistema (entradas y salidas).
        \item Puede ser \textbf{nítido} (claramente definido) o \textbf{difuso} (mal definido).
    \end{itemize}
\end{frame}

%------------------------------------------------

\begin{frame}{Componentes e Interacciones}
    \begin{columns}[c]
    \column{.5\textwidth}
        \textbf{Componentes:}
        \begin{itemize}
            \item Son los bloques de construcción fundamentales.
            \item Sus relaciones de entrada-salida se pueden modelar con ecuaciones diferenciales o en diferencias.
            \item \textbf{Ejemplo de negocio:} Clientes, proveedores, gobierno.
        \end{itemize}

    \column{.5\textwidth}
        \textbf{Interacciones:}
        \begin{itemize}
            \item Pueden ser estáticas o dinámicas.
            \item Restringidas o no restringidas.
            \item Unidireccionales o bidireccionales.
            \item Fuerza de interacción:
                \begin{itemize}
                    \item 0: Sin interacción.
                    \item 1: Interacción total.
                    \item Entre 0 y 1: Interacción parcial.
                \end{itemize}
        \end{itemize}
    \end{columns}
\end{frame}

%------------------------------------------------

\begin{frame}{El Entorno del Sistema}
    \frametitle{Entradas, Salidas y Procesamiento}
    \begin{block}{Interacción Continua}
    Los sistemas abiertos, como los organismos vivos, intercambian continuamente materia y energía con su entorno para sobrevivir.
    \end{block}
    \begin{itemize}
        \item \textbf{Entrada (Input):} Lo que entra al sistema desde el exterior.
        \item \textbf{Salida (Output):} Lo que sale del sistema hacia el entorno.
        \item \textbf{Procesamiento (Throughput):} La transformación de la entrada en salida.
        \item El entorno incluye: competencia, tecnología, capital, regulación, etc.
    \end{itemize}
\end{frame}

%------------------------------------------------

\begin{frame}{Enfoques de Estudio: Caja Negra vs. Caja Blanca}
    \begin{columns}[t]
        \column{.48\textwidth}
            \begin{alertblock}{Caja Negra (Black Box)}
                \begin{itemize}
                    \item Nos preocupamos solo por la \textbf{entrada} y la \textbf{salida}.
                    \item Se ignoran las complejidades internas del sistema.
                    \item \textbf{Ejemplo:} Medir el consumo total de combustible (entrada) de una ciudad y su nivel de emisiones (salida) sin detallar el consumo individual.
                \end{itemize}
            \end{alertblock}

        \column{.48\textwidth}
            \begin{block}{Caja Blanca (White Box)}
                \begin{itemize}
                    \item Nos preocupamos por los \textbf{detalles internos} y los procesos del sistema, además de las entradas y salidas.
                    \item \textbf{Ejemplo:} Rastrear el movimiento de cada tanque de combustible a cada edificio particular de la ciudad para entender la contaminación.
                \end{itemize}
            \end{block}
    \end{columns}
\end{frame}

%------------------------------------------------

%------------------------------------------------
\section{Clasificación de Sistemas}
%------------------------------------------------

\begin{frame}{Clasificación de Sistemas}
    \frametitle{Dimensiones para Clasificar Sistemas}
    \begin{block}{Importancia}
        Comprender las diversas clasificaciones de los sistemas es esencial para seleccionar la metodología de modelado y simulación más apropiada para un problema dado.
    \end{block}
    \begin{itemize}
        \item Según el marco temporal.
        \item Según la complejidad.
        \item Según el nivel de incertidumbre.
        \item Según la varianza en el tiempo.
        \item Linealidad y no linealidad.
        \item Varianza e invarianza en el tiempo.
    \end{itemize}
\end{frame}

%------------------------------------------------
\subsection{Según el marco temporal}
%------------------------------------------------

\begin{frame}{Sistemas Continuos vs. Discretos}
    \frametitle{Clasificación según el Marco Temporal}
    \begin{columns}[t]
        \column{.48\textwidth}
            \begin{block}{Sistemas Continuos}
                \begin{itemize}
                    \item Las variables de estado cambian de manera \textbf{continua} e ininterrumpida.
                    \item Descritos por \alert{ecuaciones diferenciales}.
                    \item \textbf{Ejemplo:} El nivel de agua en un tanque que se llena y vacía, el vuelo de un misil.
                \end{itemize}
            \end{block}

        \column{.48\textwidth}
            \begin{alertblock}{Sistemas Discretos}
                 \begin{itemize}
                    \item Las variables de estado cambian solo en puntos \textbf{discretos} en el tiempo (eventos).
                    \item El estado permanece constante entre eventos.
                    \item \textbf{Ejemplo:} Un sistema de colas en un banco, sistemas logísticos y de manufactura.
                \end{itemize}
            \end{alertblock}
    \end{columns}
    \vspace{1em}
    \begin{block}{Sistemas Híbridos}
    Contienen componentes tanto continuos como discretos. Ejemplo: una planta química con flujo continuo controlado por válvulas discretas.
    \end{block}
\end{frame}

%------------------------------------------------
\subsection{Según la complejidad}
%------------------------------------------------

\begin{frame}{Clasificación por Complejidad}
    \begin{block}{Sistemas Físicos:}
        Variables medibles cuantitativamente con dispositivos (eléctricos, mecánicos, etc.). Menos complejos.
    \end{block} 
    \begin{block}{Sistemas Conceptuales:} 
    Mediciones cualitativas o imaginarias (psicológicos, sociales, económicos). Más complejos.
    \end{block}
    \begin{block}{Sistemas Esotéricos:}
    Las mediciones no son posibles con dispositivos físicos. Complejidad del más alto orden.
    \end{block}

\end{frame}

\begin{frame}{Clasificación por Complejidad}
    \begin{figure}
                \includegraphics[width=0.9\linewidth]{Figuras/Cap1/fig1.2.png}
                \caption{Clasificación de sistemas con base en su complejidad}
                \label{fig:1.2}
            \end{figure}
\end{frame}

%------------------------------------------------
\subsection{Según la incertidumbre}
%------------------------------------------------

\begin{frame}{Sistemas Determinísticos, Estocásticos y Difusos}
    \frametitle{Clasificación según el Nivel de Incertidumbre}
    \begin{block}{Determinísticos:}
        \begin{itemize}
            \item Comportamiento futuro predecible con certeza si se conoce el estado actual y las entradas. Sin componentes aleatorios.
            \item \textit{Ejemplo:} Movimiento de los planetas.
        \end{itemize}
        \end{block}

        \begin{alertblock}{Estocásticos:}
        \begin{itemize}
            \item Contienen al menos un componente aleatorio. El comportamiento no puede predecirse con certeza.
            \item \textit{Ejemplo:} Tiempos de llegada de clientes, fallos de máquinas.
        \end{itemize}
        \end{alertblock}

      \begin{block}{Difusos:}
        \begin{itemize}
            \item Caracterizados por variables cuantificables en términos lingüísticos y con alto grado de ambigüedad.
        \end{itemize}
    \end{block}
\end{frame}

\begin{frame}{Señales y Sistemas: Continuo vs. Discreto}
    \begin{columns}[c]
        \column{.5\textwidth}
            \begin{itemize}
                \item \textbf{Sistemas de Tiempo Continuo:} Entradas y salidas son señales de tiempo continuo.
                \item \textbf{Sistemas de Tiempo Discreto:} Entradas y salidas son señales definidas solo en instantes discretos.
            \end{itemize}
        \column{.45\textwidth}
            \begin{figure}
                \includegraphics[width=\linewidth]{Figuras/Cap1/fig1.6.png}
                \caption{Sistemas analógicos y digitales.}
                \label{fig:1.6}
            \end{figure}
    \end{columns}
\end{frame}

\begin{frame}{Procesamiento de Señales}
    \frametitle{Procesando Señales Continuas con Sistemas Discretos}
    \begin{block}{Conversión Analógico-Digital (ADC) y Digital-Analógico (DAC)}
    Para procesar señales continuas con sistemas discretos, es necesario convertirlas.
    \end{block}
    \begin{figure}
        \includegraphics[width=0.9\textwidth]{Figuras/Cap1/fig1.7.png}
        \caption{Procesamiento de señales de tiempo continuo por sistemas de tiempo discreto.}
        \label{fig:1.7}
    \end{figure}
\end{frame}

\begin{frame}{Tipos de Señales: Eje de Tiempo vs. Amplitud}
    \begin{alertblock}{Calificando las Señales}
    \textbf{Tiempo Discreto/Continuo:} Califica la naturaleza de la señal en el eje del tiempo (eje X). \\
    \textbf{Analógico/Digital:} Califica la naturaleza de la amplitud de la señal (eje Y).
    \end{alertblock}
\end{frame}

\begin{frame}{Tipos de Señales: Eje de Tiempo vs. Amplitud}
    
    \begin{figure}
        \includegraphics[width=0.5\textwidth]{Figuras/Cap1/fig1.8.png}
        \caption{Diferentes tipos de señales.}
        \label{fig:1.8}
    \end{figure}
\end{frame}

%------------------------------------------------
\subsection{Según la varianza en el tiempo}
%------------------------------------------------

\begin{frame}{Sistemas Estáticos vs. Dinámicos}
    \frametitle{Clasificación según la Varianza en el Tiempo}
    \begin{columns}[t]
        \column{.48\textwidth}
            \begin{block}{Sistemas Estáticos}
                \begin{itemize}
                    \item El tiempo no juega un papel explícito.
                    \item Representan el sistema en un único punto en el tiempo.
                    \item \textbf{Técnica:} Simulación de Monte Carlo.
                    \item \textit{Ejemplo:} Estimar el costo de un proyecto con costos de actividad aleatorios.
                \end{itemize}
            \end{block}

        \column{.48\textwidth}
            \begin{alertblock}{Sistemas Dinámicos}
                 \begin{itemize}
                    \item El estado del sistema evoluciona con el tiempo.
                    \item La gran mayoría de los sistemas de interés.
                    \item \textit{Ejemplos:} Sistemas de colas (discreto), nivel de un tanque (continuo).
                \end{itemize}
            \end{alertblock}
    \end{columns}
\end{frame}

%------------------------------------------------
\subsection{Sistemas Lineales y No Lineales}
%------------------------------------------------

\begin{frame}{Sistemas Lineales y No Lineales}
    \begin{columns}[c]
    \column{.45\textwidth}
     \textbf{Importancia de los Sistemas Lineales:}
        \begin{enumerate}
            \item La mayoría de las situaciones de ingeniería son lineales (en un rango).
            \item La mayoría de las situaciones en ciencias sociales no son lineales.
            \item Existen soluciones exactas mediante técnicas estándar.
        \end{enumerate}
    \textbf{Sistemas No Lineales:}
    \begin{itemize}
        \item No hay métodos estándar para analizarlos.
        \item Requieren enfoques gráficos, experimentales o aproximaciones.
    \end{itemize}
    \column{.45\textwidth}
     \begin{figure}
        \includegraphics[width=\linewidth]{Figuras/Cap1/fig1.3.png}
        \caption{Representación de un sistema de dos puertos.}
        \label{fig:1.3}
    \end{figure}
    \end{columns}
\end{frame}

\begin{frame}{Linealidad en la Práctica: Ley de Ohm y Ley de Hooke}
    \begin{block}{Ley de Ohm}
    La relación lineal ($V \propto I$) no se mantiene siempre. Si la corriente aumenta excesivamente, la resistencia cambia por la temperatura:
    \begin{equation*}
        R_t = R_0(1 + \alpha \Delta T)
    \end{equation*}
    \end{block}

    \begin{alertblock}{Ley de Hooke}
    La relación lineal (Esfuerzo $\propto$ Deformación) se rompe cuando el esfuerzo excede el límite elástico del material.
    \end{alertblock}
\end{frame}


\begin{frame}{Principios de los Sistemas Lineales}
    \frametitle{Superposición y Homogeneidad}
    Un sistema lineal obedece los teoremas de superposición y homogeneidad.
    \begin{block}{Teorema de Superposición}
     Si $e_1(t) \rightarrow \omega_1(t)$ y $e_2(t) \rightarrow \omega_2(t)$, entonces:
     \begin{equation*}
     e_1(t) + e_2(t) \rightarrow \omega_1(t) + \omega_2(t)
     \end{equation*}
     Una superposición de excitaciones resulta en una superposición de respuestas.
    \end{block}
 \begin{Example}
 La relación entre venta y utilidad en una gasolinera
 \begin{itemize}
     \item Si vende 100 lts de gasolina gana \$400 M.N.
     \item Si vende 500 lts de gasolina gana \$2000 M.N.
     \item Entonces, si vende 500 + 100 lts de gasolina gana \$2000 + 400 M.N.
 \end{itemize}
 \end{Example}
\end{frame}

\begin{frame}{Principios de los Sistemas Lineales}
    \frametitle{Superposición y Homogeneidad}
\begin{alertblock}{Homogeneidad}
    Si se aplican $n$ excitaciones idénticas $e_1(t)$, entonces:
     \begin{equation*}
     n \cdot e_1(t) \rightarrow n \cdot \omega_1(t)
     \end{equation*}
     La magnitud de la respuesta es proporcional a la magnitud de la excitación.
    \end{alertblock}
    \begin{Example}
 La relación entre venta y utilidad en una gasolinera
 \begin{itemize}
     \item Si vende 100 lts de gasolina gana \$400 M.N.
     \item Entonces, si vende 5 x 100 lts de gasolina gana \$5 x 400 M.N.
 \end{itemize}
 \end{Example}
\end{frame}
\begin{frame}{Definición Matemática de Sistemas Lineales}
    \frametitle{Ecuaciones Diferenciales}
    \begin{block}{Ecuación Diferencial Lineal}
        Una ecuación diferencial ordinaria de segundo orden lineal:
        \begin{equation*}
            \frac{d^2\omega}{dt^2} + a_1\frac{d\omega}{dt} + a_0\omega = e(t)
        \end{equation*}
        Es lineal porque ni $\omega$ ni sus derivadas están elevadas a una potencia o multiplicadas entre sí.
    \end{block}

    \begin{alertblock}{Ecuaciones Diferenciales No Lineales}
        Se vuelven no lineales si hay:
        \begin{itemize}
            \item Un producto de la variable dependiente y su derivada: $y\frac{dy}{dt}$
            \item Una potencia de la variable dependiente: $u^2$
            \item Una potencia de una derivada: $\left(\frac{d^2y}{dt^2}\right)^2$
        \end{itemize}
    \end{alertblock}
\end{frame}

%------------------------------------------------
\subsection{Sistemas Variables vs. Invariantes en el Tiempo}
%------------------------------------------------

\begin{frame}{Sistemas Variables vs. Invariantes en el Tiempo}
    \begin{columns}[c]
        \column{.48\textwidth}
            \begin{block}{Sistemas Variables en el Tiempo}
                \begin{itemize}
                    \item Sus parámetros cambian con el tiempo.
                    \item \textbf{Ejemplo:} Un micrófono de carbón (la resistencia cambia con la presión), el rendimiento de combustible de un auto a lo largo de su vida.
                \end{itemize}
            \end{block}

        \column{.48\textwidth}
            \begin{figure}
                \includegraphics[width=\linewidth]{Figuras/Cap1/fig1.4.png}
                \caption{Respuestas de un sistema invariante en el tiempo.}
                \label{fig:1.4}
            \end{figure}
    \end{columns}
\end{frame}

\begin{frame}{Sistemas Variables vs. Invariantes en el Tiempo}
    \begin{columns}[c]
        \column{.48\textwidth}
             \begin{alertblock}{Sistemas Invariantes en el Tiempo}
                 \begin{itemize}
                    \item Sus parámetros no cambian con el tiempo.
                    \item La salida depende de la forma de la entrada, no del instante en que se aplica.
                    \item Si $e(t) \rightarrow \omega(t)$, entonces $e(t - \tau) \rightarrow \omega(t - \tau)$.
                \end{itemize}
            \end{alertblock}

        \column{.48\textwidth}
            \begin{figure}
                \includegraphics[width=\linewidth]{Figuras/Cap1/fig1.4.png}
                \caption{Respuestas de un sistema invariante en el tiempo.}
                \label{fig:1.4}
            \end{figure}
    \end{columns}
\end{frame}
%----------------------------------------------------------------------------------------
%------------------------------------------------
\section{La Filosofía de Sistemas}
%------------------------------------------------

\begin{frame}{La Filosofía de Sistemas}
    \begin{block}{Teoría General de Sistemas}
        Postulada por \textbf{Ludwig Von Bertalanffy}, aboga por la existencia de una ley general de sistemas aplicable tanto a sistemas físicos como humanos y sociales, integrando ciencia, artes, ética y política.
    \end{block}
    \begin{itemize}
        \item Surge cuando la ciencia tradicional no lograba explicar los \textbf{sistemas abiertos}.
        \item Se basa en las analogías formales entre sistemas físicos, químicos y biológicos.
        \item Busca una comprensión holística del mundo.
    \end{itemize}
\end{frame}

%------------------------------------------------

\begin{frame}{Karl Ludwig von Bertalanffy}
    \frametitle{(1901-1972)}
        \begin{columns}[c]
        \column{.6\textwidth}
                \begin{alertblock}{Padre del Pensamiento Sistémico Moderno}
                 Biólogo austriaco, fundador de la 'teoría general de sistemas' o 'sistemología general'.
                \end{alertblock}
                \begin{itemize}
                    \item Sus contribuciones se extendieron a la cibernética, educación, filosofía, psicología y sociología.
                    \item Su modelo de crecimiento individual (1934) es ampliamente utilizado en modelos biológicos.
                    \item Su contribución más aclamada es la \textbf{teoría de los sistemas abiertos}.
                \end{itemize}
        \column{.4\textwidth}
            \begin{figure}
                \includegraphics[width=0.8\linewidth]{Figuras/Cap1/Bertalanffy.jpeg} % Sugerencia de imagen, puedes cambiarla
                \caption{Karl Ludwig von Bertalanffy.}
            \end{figure}
    \end{columns}
\end{frame}

%------------------------------------------------

\begin{frame}{De Aristóteles a Newton}
    \frametitle{Dos Visiones del Mundo}
     \begin{columns}[t]
        \column{.48\textwidth}
            \begin{block}{Visión Holística de Aristóteles}
                \begin{itemize}
                    \item 'Un todo es más que la suma de sus partes.'
                    \item Visión \textbf{teleológica}: las cosas suceden para cumplir su propósito interno.
                    \item Base conceptual de la filosofía de sistemas.
                \end{itemize}
            \end{block}

        \column{.48\textwidth}
            \begin{alertblock}{Revolución Científica (S. XVII)}
                 \begin{itemize}
                    \item \textbf{Copérnico, Kepler, Galileo, Newton}.
                    \item Demolió la visión aristotélica.
                    \item Propuso una imagen \textbf{mecánica} del universo, expresada en lenguaje matemático.
                    \item La perspectiva teleológica pareció innecesaria.
                \end{itemize}
            \end{alertblock}
    \end{columns}
\end{frame}

%------------------------------------------------
\subsection{El Método de la Ciencia}
%------------------------------------------------

\begin{frame}{El Método Científico: Las 3R}
    \frametitle{¿Cómo se adquiere conocimiento científicamente?}
    \begin{block}{Definición}
    La ciencia adquiere conocimiento públicamente comprobable mediante el pensamiento racional aplicado a la observación y experimentación.
    \end{block}

    Se caracteriza por tres pilares fundamentales, las 3R:
    \begin{columns}[c]
        \column{.3\textwidth}
        \begin{center}
        \huge \alert{R}
        \end{center}
        \textbf{Reduccionismo}

        \column{.3\textwidth}
        \begin{center}
        \huge \alert{R}
        \end{center}
        \textbf{Repetibilidad}

        \column{.3\textwidth}
        \begin{center}
        \huge \alert{R}
        \end{center}
        \textbf{Refutación}
    \end{columns}
\end{frame}


\begin{frame}{Explicando las 3R}
 
        \begin{block}{Reduccionismo:}
            \begin{itemize}
                \item Seleccionar solo algunos elementos para investigar.
                \item Descomponer problemas complejos y analizarlos por partes.
                \item Aceptar la explicación más simple requerida por los hechos.
            \end{itemize}
       \end{block}
        
        \begin{block}{Repetibilidad:}
            \begin{itemize}
                \item La salida de un sistema es la misma para una excitación dada, sin importar el tiempo o lugar.
                \item Hace que el conocimiento sea \alert{público} y verificable por cualquiera.
            \end{itemize}
        \end{block}
         \begin{block}{Refutación:}
            \begin{itemize}
                \item El progreso intelectual se produce al intentar refutar hipótesis audaces.
                \item No conformarse, sino desafiar el paradigma existente.
            \end{itemize}
\end{block}
\end{frame}

%------------------------------------------------
\subsection{Problemas de la Ciencia y el Surgimiento del Sistema}
%------------------------------------------------

\begin{frame}{Limitaciones de la Ciencia}
    \frametitle{¿Por qué necesitamos la Filosofía de Sistemas?}
    \begin{alertblock}{La Debilidad del Reduccionismo}
    El reduccionismo, aunque es una herramienta poderosa, también es la principal debilidad del método científico tradicional.
    \end{alertblock}
    \begin{itemize}
        \item La ciencia asume que los componentes de un todo son los mismos cuando se examinan de forma aislada.

        \item Esto es cierto para sistemas físicos \textbf{regulares y bien estructurados}.

        \item Sin embargo, \textbf{falla} al estudiar fenómenos complejos como la sociedad humana o los sistemas biológicos, donde las interacciones son cruciales.

        \item Es en este contexto que Von Bertalanffy retoma la noción holística de Aristóteles: \textbf{'Un todo es más que la suma de sus partes'}.
    \end{itemize}
\end{frame}
%------------------------------------------------
\section{Modelado de Sistemas}
%------------------------------------------------

\begin{frame}{Modelado de Sistemas - Introducción}
    \begin{block}{El Modelado en la Vida Cotidiana}
    El ser humano ha modelado desde que desarrolló la capacidad de imaginar. Un modelo es una representación o abstracción de un sistema real.
    \end{block}
    \begin{itemize}
        \item Un niño con una muñeca.
        \item Un arquitecto con una maqueta.
        \item Un empresario con un plan de negocios.
    \end{itemize}
    \begin{alertblock}{Objetivo del Modelado}
    El modelado nos permite entender, predecir y analizar sistemas complejos sin necesidad de interactuar directamente con ellos.
    \end{alertblock}
\end{frame}

%------------------------------------------------

\begin{frame}{Objetivos del Modelado}
    \frametitle{¿Qué problemas podemos resolver?}
    \begin{columns}[t]
        \column{.5\textwidth}
        Un modelo nos ayuda a:
            \begin{itemize}
                \item Medir la altura de una torre.
                \item Medir el ancho de un río.
                \item Calcular la masa de la Tierra.
                \item Estimar la temperatura del sol.
                \item Predecir el rendimiento de un cultivo.
                \item Cuantificar la sangre en un cuerpo.
            \end{itemize}
        \column{.5\textwidth}
            \begin{itemize}
                \item Proyectar la población futura.
                \item Determinar la órbita de un satélite.
                \item Evaluar el impacto de políticas económicas.
                \item Optimizar el diseño de un producto.
                \item Estimar la vida útil de un componente.
                \item Pronosticar reclamaciones de seguros.
            \end{itemize}
    \end{columns}
\end{frame}

%------------------------------------------------

\begin{frame}{El Proceso de Modelado}
    \begin{block}{Abstracción de la Realidad}
    Un modelo es un marco conceptual que describe un sistema. Puede ser una réplica física o una abstracción lógica/matemática.
    \end{block}

    \begin{alertblock}{Compromiso entre Simplicidad y Precisión}
    El desarrollo de un modelo requiere un balance entre:
    \begin{enumerate}
        \item \textbf{La simplicidad del modelo:} A más suposiciones, más simple el modelo.
        \item \textbf{La precisión (fidelidad) del modelo:} A menos suposiciones, más complejo pero más preciso.
    \end{enumerate}
    La precisión de un modelo es complementaria a su simplicidad.
    \end{alertblock}
\end{frame}

%------------------------------------------------

\begin{frame}{Modelos de Caja Blanca vs. Caja Negra}
    \frametitle{Clasificación según la Información Disponible}
    \begin{columns}[t]
        \column{.48\textwidth}
            \begin{block}{Caja Blanca (White Box)}
                \begin{itemize}
                    \item Toda la información necesaria sobre el sistema está disponible.
                    \item Se conoce la relación funcional entre variables.
                    \item \textit{Ejemplo:} Modelar el efecto de un medicamento sabiendo que sigue una función de decaimiento exponencial (aunque los parámetros deban estimarse).
                \end{itemize}
            \end{block}

        \column{.48\textwidth}
            \begin{alertblock}{Caja Negra (Black Box)}
                 \begin{itemize}
                    \item No hay información previa disponible sobre el sistema.
                    \item Se busca estimar tanto la relación funcional como los parámetros.
                    \item Se usan funciones generales como las \textbf{Redes Neuronales Artificiales (ANN)}.
                \end{itemize}
            \end{alertblock}
    \end{columns}
\end{frame}

%------------------------------------------------
\subsection*{Necesidad del Modelado}
%------------------------------------------------

\begin{frame}{¿Por qué Modelar en Lugar de Experimentar?}
    \frametitle{La Necesidad del Modelado de Sistemas}
    A veces es inapropiado o imposible experimentar en sistemas reales.
    \begin{block}{Razones Principales}
        \begin{description}
            \item[\textbf{Demasiado Caro:}] La experimentación física con sistemas complejos (e.g., satélites) es extremadamente costosa y requiere mucho tiempo.
            \item[\textbf{Arriesgado:}] Existe el riesgo de dañar el sistema o, peor aún, un riesgo para la vida humana (e.g., entrenar a un operario en una planta nuclear).
        \end{description}
    \end{block}

    \begin{alertblock}{Situaciones Esenciales para el Modelado}
        \begin{itemize}
            \item Diseño de sistemas que aún no existen (e.g., un nuevo avión).
            \item Obliga a pensar con claridad sobre la estructura y elementos esenciales.
            \item Es una herramienta para mejorar la comprensión y el rendimiento del sistema.
            \item Permite explorar múltiples soluciones de forma económica.
        \end{itemize}
    \end{alertblock}
\end{frame}

%------------------------------------------------
\subsection*{Métodos de Modelado para Sistemas Complejos}
%------------------------------------------------

\begin{frame}{Modelado de Sistemas Complejos}
    \begin{block}{El Desafío de la Complejidad}
    Modelar cada detalle de un sistema complejo (como un avión de combate) es computacionalmente inviable y aumenta la incertidumbre. Es necesario hacer aproximaciones para reducir el modelo a un tamaño razonable.
    \end{block}

    \begin{alertblock}{Técnicas según la Complejidad}
    \begin{description}
        \item [Sistemas menos complejos:] Técnicas de modelado matemático.
        \item [Sistemas de complejidad media:] Redes Neuronales Artificiales (ANN).
        \item [Sistemas altamente complejos:] Modelado de sistemas difusos (Lógica Difusa).
    \end{description}
    \end{alertblock}
\end{frame}

%------------------------------------------------

\begin{frame}{Enfoques de Modelado según la Complejidad}
    \begin{figure}[h!]
        \centering
        \includegraphics[width=0.5\textwidth]{Figuras/Cap1/fig1.9.png}
        \caption{Diferentes enfoques de modelado.}
        \label{fig:1.9}
    \end{figure}
\end{frame}

%------------------------------------------------
\section{Clasificación de Modelos}
%------------------------------------------------

\begin{frame}{Clasificación de Modelos}

    \begin{block}{Utilidad de los Modelos}
    Los modelos son un medio aceptado para estudiar fenómenos complejos a un costo menor y en menos tiempo que experimentar con sistemas reales. Nos informan sobre nuestra ignorancia y mejoran la comprensión del sistema.
    \end{block}
\end{frame}

\begin{frame}{Clasificación de Modelos}

    \begin{figure}[h!]
        \centering
        \includegraphics[width=0.5\textwidth]{Figuras/Cap1/fig1.10.png}
        \caption{Representación pictórica de la clasificación de modelos.}
        \label{fig:1.10}
    \end{figure}
\end{frame}

%------------------------------------------------
\subsection*{Modelo Físico vs. Abstracto}
%------------------------------------------------

\begin{frame}{Modelo Físico vs. Abstracto}
    \begin{columns}[t]
        \column{.48\textwidth}
            \begin{block}{Modelo Físico (Icónico)}
                \begin{itemize}
                    \item Son réplicas físicas, a menudo a escala reducida.
                    \item Fáciles de entender visualmente.
                    \item \textbf{Ejemplos:}
                        \begin{itemize}
                            \item Coches de arcilla en túneles de viento.
                            \item Cabinas de entrenamiento para pilotos.
                            \item Maquetas de edificios.
                        \end{itemize}
                \end{itemize}
            \end{block}

        \column{.48\textwidth}
            \begin{alertblock}{Modelo Abstracto}
                 \begin{itemize}
                    \item Representan un sistema usando símbolos y relaciones lógicas o cuantitativas.
                    \item Son los más comunes en análisis de sistemas e investigación de operaciones.
                    \item \textbf{Ejemplo:} Un modelo matemático.
                \end{itemize}
            \end{alertblock}
    \end{columns}
\end{frame}

%------------------------------------------------
\subsection*{Modelo Matemático vs. Descriptivo}
%------------------------------------------------

\begin{frame}{Modelo Matemático vs. Descriptivo}
    \begin{block}{Modelo Matemático}
    Es una subdivisión de los modelos abstractos que utiliza el lenguaje de los símbolos matemáticos para describir un sistema.
    \end{block}
    \begin{examples}
    Un modelo matemático abstracto muy conocido es la relación:
    \begin{equation*}
        \text{Distancia} = \text{Aceleración} \times \text{Tiempo} \quad (d = a \cdot t)
    \end{equation*}
    La validez de este modelo depende completamente del contexto en el que se aplica.
    \end{examples}
\end{frame}

%------------------------------------------------
\subsection*{Modelo Estático vs. Dinámico}
%------------------------------------------------

\begin{frame}{Modelo Estático vs. Dinámico}
     \begin{columns}[t]
        \column{.48\textwidth}
            \begin{block}{Modelo Estático}
                \begin{itemize}
                    \item Representa un sistema en un \textbf{momento particular}.
                    \item El tiempo no juega un papel relevante.
                    \item \textbf{Ejemplo:} Una maqueta arquitectónica para visualizar una planta.
                \end{itemize}
            \end{block}

        \column{.48\textwidth}
            \begin{alertblock}{Modelo Dinámico}
                 \begin{itemize}
                    \item Representa un sistema a medida que \textbf{evoluciona en el tiempo}.
                    \item Trata con interacciones que varían temporalmente.
                    \item \textbf{Ejemplo:} Un modelo de un sistema de cintas transportadoras en una fábrica.
                \end{itemize}
            \end{alertblock}
    \end{columns}
\end{frame}

%------------------------------------------------
\subsection*{Otros Tipos de Modelos}
%------------------------------------------------

\begin{frame}{Más Clasificaciones de Modelos}

        \begin{block}{Estado Estacionario vs. Transitorio:}
        \begin{itemize}
            \item \textit{Estacionario:} El comportamiento es representativo y similar en cualquier período de tiempo.
            \item \textit{Transitorio:} La respuesta del sistema cambia con el tiempo (e.g., un sistema en crecimiento).
        \end{itemize}
        \end{block}
     
        \begin{block}{Abierto vs. de Retroalimentación (Cerrado):}
        \begin{itemize}
            \item \textit{Cerrado:} Genera internamente los valores de las variables mediante su interacción. Los sistemas de retroalimentación de información son de este tipo.
        \end{itemize}
        \end{block}

\end{frame}

%------------------------------------------------
\subsection*{Modelos Determinísticos vs. Estocásticos}
%------------------------------------------------

\begin{frame}{Modelos Determinísticos vs. Estocásticos}
    \begin{block}{Modelo Determinístico}
        \begin{itemize}
            \item No contiene componentes probabilísticos (aleatorios).
            \item La salida está 'determinada' una vez que se especifican las entradas y relaciones.
            \item \textbf{Ejemplo:} Un sistema de ecuaciones diferenciales que describe una reacción química.
        \end{itemize}
    \end{block}
    \begin{alertblock}{Modelo Estocástico}
        \begin{itemize}
            \item Contiene al menos un componente de entrada aleatorio.
            \item La salida es en sí misma aleatoria y debe tratarse como una estimación.
            \item \textbf{Ejemplo:} La mayoría de los sistemas de colas e inventarios.
        \end{itemize}
    \end{alertblock}
\end{frame}

%------------------------------------------------
\subsection*{Modelos Continuos vs. Discretos}
%------------------------------------------------

\begin{frame}{Modelos Continuos vs. Discretos}
    \begin{columns}[c]
        \column{.5\textwidth}
            La decisión de usar un modelo discreto o continuo depende de los \textbf{objetivos específicos} del estudio.
            \begin{itemize}
                \item \textbf{Ejemplo (Flujo de tráfico):}
                \begin{itemize}
                    \item \textit{Modelo Discreto:} Si el movimiento de coches individuales es importante.
                    \item \textit{Modelo Continuo:} Si los coches pueden tratarse 'en conjunto' mediante ecuaciones diferenciales.
                \end{itemize}
            \end{itemize}
        \column{.45\textwidth}
            \begin{figure}
                \includegraphics[width=\linewidth]{Figuras/Cap1/fig1.11.png}
                \caption{Funciones (A) Continuas y (B) Discretas.}
                \label{fig:1.11}
            \end{figure}
    \end{columns}
\end{frame}

%------------------------------------------------
\section{El Rol y el Poder de la Simulación}
%------------------------------------------------

\begin{frame}{El Rol y el Poder de la Simulación}
    \begin{block}{Un Laboratorio Virtual}
    La simulación es una de las herramientas más utilizadas en la investigación de operaciones y la ingeniería. Su poder reside en su capacidad para experimentar con un \textbf{modelo digital del sistema} en lugar de experimentar con el sistema real.
    \end{block}
    \begin{alertblock}{¿Por qué usar un laboratorio virtual?}
    Experimentar en el sistema real podría ser:
    \begin{itemize}
        \item Costoso 
        \item Peligroso 
        \item Simplemente imposible 
    \end{itemize}
    \end{alertblock}
\end{frame}

%------------------------------------------------
\subsection*{Aplicaciones Clave en Logística e Ingeniería}
%------------------------------------------------

\begin{frame}{Aplicaciones Clave en Logística e Ingeniería}
    \begin{itemize}
        \item \textbf{Diseño y análisis de sistemas de manufactura:} Determinar el número óptimo de máquinas, tamaño de buffers y analizar el impacto de fallos.
        \item \textbf{Gestión de la cadena de suministro:} Analizar el Efecto Látigo, optimizar políticas de inventario y evaluar estrategias de distribución.
        \item \textbf{Diseño de layouts de almacenes:} Optimizar flujos de materiales, determinar cantidad de muelles y planificar asignación de personal.
        \item \textbf{Sistemas de servicio al cliente:} Determinar el número de servidores (cajeros, agentes) para cumplir con un nivel de servicio.
        \item \textbf{Planificación de proyectos:} Estimar la probabilidad de completar un proyecto a tiempo y dentro del presupuesto.
    \end{itemize}
\end{frame}

%------------------------------------------------
\subsection*{Ventajas y Desventajas de la Simulación}
%------------------------------------------------

\begin{frame}{Ventajas y Desventajas de la Simulación}
    \frametitle{Fortalezas y Debilidades de la Herramienta}
    \begin{columns}[t]
        \column{.48\textwidth}
            \begin{block}{Ventajas}
                \begin{itemize}
                    \item Permite estudiar sistemas complejos sin modelos analíticos.
                    \item Permite experimentar sin interrumpir el sistema real.
                    \item Permite comprimir o expandir el tiempo.
                    \item Ayuda a identificar cuellos de botella.
                    \item Es una poderosa herramienta de comunicación y visualización.
                \end{itemize}
            \end{block}

        \column{.48\textwidth}
            \begin{alertblock}{Desventajas}
                 \begin{itemize}
                    \item Construir un buen modelo puede ser largo y costoso.
                    \item Los resultados son estimaciones estadísticas; se requieren múltiples ejecuciones.
                    \item Los resultados pueden ser difíciles de interpretar.
                    \item Es una herramienta de evaluación, no de optimización directa.
                \end{itemize}
            \end{alertblock}
    \end{columns}
\end{frame}

%------------------------------------------------
\section{Fases del Modelado y de la Simulación}
%------------------------------------------------

\begin{frame}{El Proceso de Modelado}
    \frametitle{¿Qué es Modelar?}
    \begin{block}{Definición}
        El modelado es el arte/proceso de desarrollar un modelo de sistema. Su propósito es exponer el funcionamiento interno de un sistema y presentarlo en una forma útil para su estudio. En esencia, es el \textbf{proceso de organizar el conocimiento} sobre un sistema dado.
    \end{block}
    \begin{alertblock}{Compromiso Clave}
    Existe un compromiso (\textit{trade-off}) fundamental entre la \textbf{simplicidad} del modelo, su \textbf{precisión} (fidelidad) y el \textbf{tiempo de cómputo} requerido.
    \end{alertblock}
\end{frame}

%------------------------------------------------

\begin{frame}{Entradas y Proceso de Modelado}
    \begin{columns}[c]
        \column{.5\textwidth}
            \begin{figure}
                \includegraphics[width=\linewidth]{Figuras/Cap1/fig1.12.png}
                \caption{Entradas para el desarrollo de un modelo.}
                \label{fig:1.12}
            \end{figure}
        \column{.5\textwidth}
            \begin{figure}
                \includegraphics[width=0.7\linewidth]{Figuras/Cap1/fig1.13.png}
                \caption{Proceso de modelado.}
                \label{fig:1.13}
            \end{figure}
    \end{columns}
\end{frame}

%------------------------------------------------

\begin{frame}{Diferentes Modelos para el Mismo Sistema}
    \frametitle{Ejemplo: Una Aeronave}
    \begin{columns}[c]
        \column{.5\textwidth}
            Para el mismo sistema se pueden desarrollar diferentes modelos según el propósito. Una aeronave puede modelarse como:
            \begin{enumerate}
                \item \textbf{Una partícula:} Para estudiar la trayectoria de vuelo y el consumo de combustible.

                \item \textbf{Un sistema de cuerpos rígidos:} Para analizar la estabilidad del vuelo ante perturbaciones.

                \item \textbf{Un sistema de cuerpos deformables:} Para realizar análisis de flameo (\textit{flutter}).
            \end{enumerate}
        \column{.45\textwidth}
            \begin{figure}
                \includegraphics[width=\linewidth]{Figuras/Cap1/fig1.14.png}
                \caption{Diferentes modelos de aeronave.}
                \label{fig:1.14}
            \end{figure}
    \end{columns}
\end{frame}

%------------------------------------------------

\begin{frame}[allowframebreaks]{Fases Principales de un Proyecto de Simulación}
    \begin{enumerate}
        \item \textbf{Formulación del Problema:} Definir claramente el problema, los objetivos y las métricas de desempeño.

        \item \textbf{Investigación y Recopilación de Datos:} Estudiar el sistema real para entender su lógica y recolectar datos para alimentar y validar el modelo.

        \item \textbf{Formulación del Modelo Conceptual:} Crear un 'borrador' abstracto del sistema, identificando componentes, variables y nivel de detalle.

        \item \textbf{Traducción del Modelo a un Programa:} Implementar el modelo conceptual en un software de simulación (e.g., SimEvents).

        \item \textbf{Verificación del Modelo:} ¿Construí el modelo \textit{correctamente} (el programa refleja el modelo conceptual)?

        \item \textbf{Validación del Modelo:} ¿Construí el modelo \textit{correcto} (el modelo es una representación precisa del sistema real)?
 
        \item \textbf{Diseño Experimental:} Planificar los escenarios a simular (duración, réplicas, parámetros).

        \item \textbf{Ejecución y Análisis de Resultados:} Correr las simulaciones y analizar los datos de salida con herramientas estadísticas.

        \item \textbf{Documentación y Presentación:} Documentar el modelo y los resultados, y presentar las conclusiones a los interesados.
    \end{enumerate}
\end{frame}

%------------------------------------------------

%------------------------------------------------
\section{Del Problema del Mundo Real al Modelo Conceptual}
%------------------------------------------------

\begin{frame}{Del Problema del Mundo Real al Modelo Conceptual}
    \begin{block}{Un Proceso Creativo y Desafiante}
    La transición de un problema, a menudo vagamente definido, a un modelo conceptual estructurado es una de las partes más creativas y críticas de un estudio de simulación. Requiere comunicación constante con los expertos del dominio.
    \end{block}

    \begin{alertblock}{Definiciones Clave del Modelo Conceptual}
    El modelo conceptual debe definir:
    \begin{itemize}
        \item Los \textbf{objetivos} del modelo.
        \item Las \textbf{entradas} (parámetros controlables) y las \textbf{salidas} (métricas de desempeño).
        \item Los \textbf{supuestos y simplificaciones}. El arte del modelado reside en saber qué omitir.
        \item Los \textbf{componentes}, sus atributos y las reglas lógicas que gobiernan su interacción.
    \end{itemize}
    \end{alertblock}

    
\end{frame}

\begin{frame}{Del Problema del Mundo Real al Modelo Conceptual}

\begin{examples}{Piedra Angular del Proyecto}
    Un buen modelo conceptual es la piedra angular de todo el proyecto. Un error en esta fase inicial se propagará a todas las fases posteriores.
    \end{examples}
\end{frame}
%------------------------------------------------
\section{Verificación y Validación (V\&V)}
%------------------------------------------------

\begin{frame}{Verificación y Validación (V\&V)}

    \begin{block}{Proceso Continuo}
    La V\&V es un proceso que se lleva a cabo durante todo el ciclo de vida del proyecto. Su objetivo es asegurar que el modelo sea creíble y tenga conexión con la realidad.
    \end{block}

    \begin{columns}[t]
        \column{.48\textwidth}
            \begin{alertblock}{Verificación}
                 Se enfoca en la relación entre el modelo conceptual y el programa.
                 \vspace{1em}
                 \textbf{Pregunta clave:} ¿Construí el modelo \textit{correctamente}? (Sin errores de programación).
            \end{alertblock}

        \column{.48\textwidth}
            \begin{alertblock}{Validación}
                Se enfoca en la relación entre el modelo de simulación y el sistema real.
                \vspace{1em}
                \textbf{Pregunta clave:} ¿Construí el modelo \textit{correcto}? (Representa la realidad para nuestros propósitos).
            \end{alertblock}
    \end{columns}
\end{frame}

%------------------------------------------------

\begin{frame}{El Triángulo del Modelado}
    \begin{figure}[h!]
        \centering
        \includegraphics[width=0.9\textwidth]{Figuras/Cap1/VtoV.png}
        \caption{El triángulo del modelado: la relación entre el sistema real, el modelo conceptual y el modelo computacional, ilustrando los dominios de la Validación y la Verificación.}
        \label{fig:v_y_v}
    \end{figure}
\end{frame}

%------------------------------------------------
\subsection*{Técnicas de Verificación y Validación}
%------------------------------------------------

\begin{frame}{Técnicas de Verificación}
    \begin{block}{Asegurando que el código sea correcto}
        \begin{itemize}
            \item Revisión del código por pares (otros programadores).
            \item Uso de un depurador (debugger) para seguir la ejecución paso a paso.
            \item Ejecución del modelo bajo condiciones simplificadas con resultados analíticos conocidos (e.g., comparar con fórmulas de un sistema M/M/1).
            \item Uso de trazas de eventos para seguir la "vida" de una entidad a través del modelo.
        \end{itemize}
    \end{block}
\end{frame}

%------------------------------------------------

\begin{frame}{Técnicas de Validación}
    \begin{alertblock}{Asegurando que el modelo sea el correcto}
        \begin{itemize}
            \item \textbf{Validación por Expertos:} Consultar a quienes conocen el sistema real si el comportamiento del modelo es razonable.
            \item \textbf{Validación con Datos Históricos:} Comparar las salidas del modelo con las salidas reales del sistema en el pasado.
            \item \textbf{Prueba de Turing:} Si un experto no puede distinguir consistentemente entre los datos de salida del modelo y los del sistema real, el modelo pasa la prueba.
            \item \textbf{Pruebas de Sensibilidad:} Analizar cómo cambian los resultados del modelo al variar los parámetros de entrada.
        \end{itemize}
    \end{alertblock}
\end{frame}

%------------------------------------------------
\end{document}