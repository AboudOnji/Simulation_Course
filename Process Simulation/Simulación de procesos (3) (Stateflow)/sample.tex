%----------------------------------------------------------------------------------------
%	PACKAGES AND THEMES
%----------------------------------------------------------------------------------------

\documentclass[aspectratio=169,xcolor=dvipsnames]{beamer}
\usetheme{SimpleDarkBlue}

\usepackage[spanish]{babel}
\usepackage{hyperref}
\usepackage{graphicx} % Allows including images
\usepackage{booktabs} % Allows the use of \toprule, \midrule and \bottomrule in tables
\usepackage{amsmath}
\usepackage{lettrine}
\usepackage{rotating}
\usepackage[dvipsnames,svgnames,x11names]{xcolor}% Para definir y usar colores (ej. en hipervínculos)
\usepackage{xurl}
\usepackage{hyperref}       % Para crear hipervínculos internos y externos
\hypersetup{
    colorlinks=true,        % Colorear los enlaces en lugar de usar recuadros
    linkcolor=blue,     % Color para enlaces internos (índice, referencias cruzadas)
    filecolor=blue, % Color para enlaces a archivos locales
    urlcolor=blue,      % Color para URLs
    citecolor=blue,     % Color para citas bibliográficas
}
\setbeamertemplate{caption}[numbered]
%----------------------------------------------------------------------------------------

\usepackage{listings}
\usepackage{xcolor} % Para colores en listings
 \definecolor{codegreen}{rgb}{0,0.6,0}
 \definecolor{codegray}{rgb}{0.5,0.5,0.5}
 \definecolor{codepurple}{rgb}{0.58,0,0.82}
 \definecolor{backcolour}{rgb}{0.97,0.97,0.99}

\lstdefinestyle{MATLABStyle}{
  language=Matlab,
  basicstyle=\ttfamily\footnotesize,
  keywordstyle=\color{blue}\bfseries,
  commentstyle=\color{codegreen},
  stringstyle=\color{violet},
  numberstyle=\tiny\color{gray},
  breakatwhitespace=false,
  breaklines=true,
  captionpos=b,
  keepspaces=true,
  numbers=left,
  numbersep=5pt,
  showspaces=false,
  showstringspaces=false,
  showtabs=false,
  tabsize=2,
  frame=lines, % Añade un marco alrededor del código
  framerule=0.4pt, % Grosor del marco
  backgroundcolor=\color{backcolour} % Color de fondo suave
}
\lstset{style=MATLABStyle}
%	TITLE PAGE
%----------------------------------------------------------------------------------------

\title{Simulación de Máquinas Finitas (Stateflow)}
\subtitle{Materia: Simulación de procesos}

\author{Prof. D.Sc. BARSEKH-ONJI Aboud}

\institute
{
    Facultad de Ingeniería \\
    Universidad Anáhuac México % Your institution for the title page
}
\date{\today} % Date, can be changed to a custom date

%----------------------------------------------------------------------------------------
%	PRESENTATION SLIDES
%----------------------------------------------------------------------------------------
% Poner esto en el preámbulo
\AtBeginSection[]
{
  \begin{frame}{Agenda}
    \tableofcontents[currentsection]
  \end{frame}
}
\begin{document}

\begin{frame}
    % Print the title page as the first slide
    \titlepage
\end{frame}

%------------------------------------------------
\section{Introducción a los Sistemas Reactivos y las Máquinas de Estados}
%------------------------------------------------

\begin{frame}{Sistemas Reactivos}
    \begin{block}{¿Qué es un Sistema Reactivo?}
    Es un sistema cuyo comportamiento está dominado por su \textbf{interacción con el entorno}. No ejecuta un algoritmo de principio a fin, sino que reacciona a una secuencia de eventos que pueden ocurrir en cualquier momento y en cualquier orden.
    \end{block}
    
    \begin{alertblock}{Ejemplos}
        \begin{itemize}
            \item El controlador de un robot industrial (agarrar, soltar, esperar).
            \item La lógica de una transacción en línea.
            \item Un sistema de control de vuelo.
        \end{itemize}
    \end{alertblock}
\end{frame}

%------------------------------------------------

\begin{frame}{Máquinas de Estados Finitos (FSM)}
    \frametitle{Finite State Machines (FSM)}
    \begin{block}{Definición}
    Una FSM es un modelo matemático de computación que se puede encontrar en exactamente uno de un número finito de \textbf{estados} en un momento dado. Puede cambiar de un estado a otro en respuesta a \textbf{eventos}; este cambio se denomina \textbf{transición}.
    \end{block}

    Una FSM se define por:
    \begin{itemize}
        \item Un conjunto de \textbf{estados} posibles.
        \item Un \textbf{estado inicial}.
        \item Un conjunto de \textbf{eventos} (o entradas).
        \item Un conjunto de \textbf{acciones} (o salidas).
        \item Una \textbf{función de transición} que define el siguiente estado.
    \end{itemize}
\end{frame}

%------------------------------------------------

\begin{frame}{Limitaciones de las FSM Clásicas}
    \begin{alertblock}{El Problema de la 'Explosión de Estados'}
    A medida que un sistema crece, el número de estados necesarios para describir todas las posibles condiciones puede volverse inmanejable.
    \end{alertblock}

    \begin{itemize}
        \item Los diagramas se vuelven planos, extensos y difíciles de entender.
        \item Las FSM clásicas carecen de mecanismos eficientes para representar:
            \begin{itemize}
                \item La \textbf{jerarquía} (estados dentro de otros estados).
                \item La \textbf{concurrencia} (procesos que ocurren al mismo tiempo).
            \end{itemize}
    \end{itemize}
\end{frame}

%------------------------------------------------
\section{Stateflow: Evolución de las Máquinas de Estados}
%------------------------------------------------

\begin{frame}{Stateflow: La Evolución de las FSM}
    \begin{block}{De los Statecharts a Stateflow}
    Para superar las limitaciones de las FSM, David Harel introdujo los \textbf{statecharts}, un lenguaje visual que extiende las máquinas de estados. \textbf{Stateflow} es la implementación de MathWorks de este poderoso formalismo, integrado completamente en Simulink.
    \end{block}
    
    \begin{alertblock}{Características Clave de Stateflow}
    Stateflow añade a las FSM conceptos cruciales como:
        \begin{itemize}
            \item Jerarquía (Hierarchy)
            \item Paralelismo (Parallelism / Orthogonality)
            \item Historial (History Junctions)
            \item Comunicación con Eventos y Datos
        \end{itemize}
    \end{alertblock}
\end{frame}

%------------------------------------------------

\begin{frame}{Características Clave: Jerarquía y Paralelismo}
    \begin{columns}[t]
        \column{.48\textwidth}
            \begin{block}{Jerarquía (Hierarchy)}
                \begin{itemize}
                    \item Los estados pueden contener otros estados (subestados), creando \textbf{superestados}.
                    \item Permite organizar la lógica de manera modular y por niveles de abstracción.
                    \item \textbf{Ejemplo:} Un estado \texttt{Procesando\_Orden} puede contener subestados como \texttt{Verificando\_Pago} y \texttt{Asignando\_Inventario}.
                \end{itemize}
            \end{block}

        \column{.48\textwidth}
            \begin{alertblock}{Paralelismo (Parallelism)}
                 \begin{itemize}
                    \item Un superestado puede tener múltiples subestados activos \textbf{simultáneamente} (estados AND).
                    \item Fundamental para modelar sistemas con modos de operación independientes y concurrentes.
                    \item \textbf{Ejemplo:} En un vehículo, el sistema de transmisión y el de climatización operan en paralelo.
                \end{itemize}
            \end{alertblock}
    \end{columns}
\end{frame}

%------------------------------------------------

\begin{frame}{Características Clave: Historial y Datos}
    \begin{columns}[t]
        \column{.48\textwidth}
            \begin{block}{Historial (History Junctions)}
                \begin{itemize}
                    \item Permite que un superestado 'recuerde' el último subestado que estuvo activo antes de una transición de salida.
                    \item Al reingresar al superestado, se puede reanudar la ejecución desde donde se quedó.
                    \item Muy útil para modelar interrupciones.
                \end{itemize}
            \end{block}

        \column{.48\textwidth}
            \begin{alertblock}{Eventos y Datos}
                 \begin{itemize}
                    \item Los diagramas de Stateflow se comunican con el modelo de Simulink a través de:
                        \begin{itemize}
                            \item \textbf{Entradas:} Eventos y datos que activan transiciones y modifican la lógica.
                            \item \textbf{Salidas:} Señales y llamadas a funciones que afectan al resto del sistema.
                        \end{itemize}
                \end{itemize}
            \end{alertblock}
    \end{columns}
\end{frame}

%------------------------------------------------
\section{El Formalismo Visual de los Statecharts de Harel}
%------------------------------------------------

\begin{frame}{La Crisis de las Máquinas de Estados Convencionales}
    \begin{block}{El Problema: La 'Explosión de Estados'}
    David Harel se propuso superar las limitaciones de las FSM tradicionales, que fallaban al describir sistemas complejos del mundo real. El principal problema era la falta de estructura, que llevaba a diagramas planos, masivos y visualmente caóticos.
    \end{block}
    
    \begin{alertblock}{La Solución de Harel}
    \begin{center}
     \textbf{statecharts = state-diagrams + depth + orthogonality + broadcast-communication}
    \end{center}
    \end{alertblock}
\end{frame}

%------------------------------------------------
\subsection{Profundidad: Jerarquía, Agrupamiento y Refinamiento}
%------------------------------------------------

\begin{frame}{Profundidad: Jerarquía y Agrupamiento}
    La primera innovación clave es la \textbf{profundidad}, que introduce el concepto de jerarquía en los estados.
    \begin{columns}[c]
        \column{0.5\textwidth}
            \begin{block}{Agrupamiento (Clustering)}
                Si varios estados tienen una transición común bajo el mismo evento, podemos agruparlos en un \textbf{superestado}. La transición se dibuja una sola vez desde el borde del superestado, economizando drásticamente el diagrama.
            \end{block}
        \column{0.45\textwidth}
            \begin{figure}
                \includegraphics[width=0.8\linewidth]{Figuras/Cap10/fig10.2H.png}
                \caption{Agrupamiento de estados con una transición común.}
                \label{fig:H2}
            \end{figure}
    \end{columns}
\end{frame}

%------------------------------------------------

\begin{frame}{Profundidad: Entradas por Defecto e Historial}
    \begin{columns}[c]
        \column{0.5\textwidth}
            \begin{alertblock}{Entrada por Defecto}
            Especifica a cuál de los subestados se debe entrar si la transición de entrada no lo indica explícitamente.
            \begin{figure}
                \includegraphics[width=0.7\linewidth]{Figuras/Cap10/fig10.6H.png}
                \caption{Transición por defecto.}
                \label{fig:H6}
            \end{figure}
            \end{alertblock}
            
        \column{0.5\textwidth}
            \begin{block}{Conector de Historial (H)}
             Permite que un superestado 'recuerde' el último subestado activo. Al reingresar, la ejecución se reanuda donde se quedó, lo cual es ideal para modelar interrupciones.
            \begin{figure}
                \includegraphics[width=0.8\linewidth]{Figuras/Cap10/fig10.10H.png}
                \caption{Conector de Historial.}
                \label{fig:H10}
            \end{figure}
            \end{block}
    \end{columns}
\end{frame}

%------------------------------------------------

\begin{frame}{Ejemplo Práctico: Alarma de Reloj Digital}
    \begin{columns}[c]
        \column{0.5\textwidth}
            En este ejemplo de Harel, los estados de sonido de la alarma se agrupan en el superestado \texttt{alarms-beep}.
            
            \begin{block}{Ventaja de la Jerarquía}
            En lugar de dibujar múltiples flechas de salida (una desde cada subestado), se dibuja una única transición desde el borde del superestado. La transición \texttt{any button pressed} se aplica sin importar en cuál de los subestados de alarma se encuentre el sistema, simplificando enormemente el diagrama.
            \end{block}
        \column{0.5\textwidth}
            \begin{figure}
                \includegraphics[width=\linewidth]{Figuras/Cap10/fig10.8H.png}
                \caption{Agrupamiento de estados de alarma en un superestado.}
                \label{fig:harel_8}
            \end{figure}
    \end{columns}
\end{frame}

%------------------------------------------------

\begin{frame}{Diagrama Completo: Reloj Citizen según Harel}
    \begin{figure}
        \centering
        \includegraphics[height=1.35\textheight, angle=-90]{Figuras/Cap10/fig10.9H.png}
        \caption{Diagrama de estado del sistema Citizen Quartz Multi-Alarm III.}
        \label{fig:harel_9}
    \end{figure}
\end{frame}

%------------------------------------------------
\section{Componentes Fundamentales de un Diagrama de Stateflow}
%------------------------------------------------

\begin{frame}{Componentes Fundamentales: Estado y Transición}
   
            \begin{block}{Estado (State)}
            Representado por un rectángulo con esquinas redondeadas, es un modo de operación del sistema. Puede tener acciones asociadas:
                \begin{itemize}
                    \item \texttt{entry (en)}: Se ejecuta una vez al entrar al estado.
                    \item \texttt{during (du)}: Se ejecuta en cada paso de tiempo mientras el estado está activo.
                    \item \texttt{exit (ex)}: Se ejecuta una vez al salir del estado.
                \end{itemize}
            \end{block}
            \begin{figure}
                \includegraphics[width=0.6\linewidth]{Figuras/Cap10/fig10.1.png}
                \caption{Diagrama simple de dos estados.}
                \label{fig:enter-label}
            \end{figure}


            

\end{frame}

%------------------------------------------------
\begin{frame}{Componentes Fundamentales: Estado y Transición}
\begin{alertblock}{Transición (Transition)}
            Representada por una flecha, indica un posible cambio de estado. Su etiqueta define la lógica:
            \begin{center}
            \texttt{evento[condición]\{acción\}/acción}
            \end{center}
                \begin{itemize}
                    \item \texttt{evento}: El suceso que 'despierta' la transición.
                    \item \texttt{[condición]}: Expresión booleana que debe ser verdadera.
                    \item \texttt{\{acción\}}: Se ejecuta si la condición es verdadera.
                    \item \texttt{/acción}: Se ejecuta al completar la transición.
                \end{itemize}
            \end{alertblock}
\end{frame}
\begin{frame}{Componentes Fundamentales: Flujo y Datos}
    \begin{columns}[t]
        \column{0.5\textwidth}
            \begin{block}{Elementos de Flujo}
                \begin{itemize}
                    \item \textbf{Transición por Defecto:} Una flecha que indica cuál es el primer estado que se activa en un nivel jerárquico.
                    \item \textbf{Conector de Unión (Junction):} Un círculo que sirve como punto de decisión para crear lógica condicional (if-then-else) en las transiciones.
                \end{itemize}
            \end{block}
        
        \column{0.5\textwidth}
            \begin{alertblock}{Interfaz del Diagrama}
                 \begin{itemize}
                    \item \textbf{Datos (Data):} Son las variables del diagrama (Input, Output, Local, Parámetros).
                    \item \textbf{Eventos (Events):} Son disparadores explícitos e instantáneos que notifican que 'algo ha sucedido'.
                \end{itemize}
            \end{alertblock}
    \end{columns}
\end{frame}

%------------------------------------------------
\section{Ejemplo de Aplicación: El Ciclo de Vida de una Orden de Compra}
%------------------------------------------------

\begin{frame}{Ejemplo: Ciclo de Vida de una Orden}
    \frametitle{Modelando el Proceso de \textit{Order Fulfillment}}
    \begin{block}{La Columna Vertebral del Negocio}
    El proceso de cumplimiento de pedidos es la secuencia de pasos desde que se recibe un pedido hasta que el producto es entregado. Su eficiencia y precisión tienen un impacto directo en la satisfacción del cliente y los costos operativos.
    \end{block}
\end{frame}

%------------------------------------------------
\subsection{Contexto y Relevancia en B2C y B2B}
%------------------------------------------------

\begin{frame}{Relevancia en B2C vs. B2B}
    \begin{columns}[t]
        \column{.48\textwidth}
            \begin{block}{Contexto B2C (e-commerce)}
                \begin{itemize}
                    \item La \textbf{experiencia del cliente} es primordial.
                    \item Se esperan visibilidad total del pedido y entregas rápidas.
                    \item El principal desafío es gestionar un \textbf{alto volumen} de pedidos pequeños y personalizados.
                    \item Un fallo puede resultar en la pérdida de un cliente para siempre.
                \end{itemize}
            \end{block}

        \column{.48\textwidth}
            \begin{alertblock}{Contexto B2B (Negocio a Negocio)}
                 \begin{itemize}
                    \item Las transacciones son de \textbf{mayor valor} y complejidad.
                    \item La \textbf{fiabilidad y la precisión} son cruciales.
                    \item Un retraso puede detener la línea de producción de un cliente, generando costos enormes.
                    \item La integración con sistemas ERP es fundamental.
                \end{itemize}
            \end{alertblock}
    \end{columns}
\end{frame}

%------------------------------------------------
\subsection{¿Por qué modelar este proceso con Stateflow?}
%------------------------------------------------

\begin{frame}{¿Por qué modelar este proceso con Stateflow?}
    \begin{block}{Naturaleza Basada en Estados y Eventos}
    Una orden de compra no es un simple dato que fluye; es una entidad que posee un \textbf{estado} en todo momento ('Pendiente', 'Procesando', 'Enviado', 'Cancelado'). Este estado cambia en respuesta a \textbf{eventos} discretos ('pago\_confirmado', 'inventario\_insuficiente'). Esta naturaleza hace que Stateflow sea la herramienta ideal.
    \end{block}
    
    \begin{alertblock}{Con Stateflow, podemos:}
        \begin{itemize}
            \item Representar visualmente todos los estados posibles de una orden.
            \item Definir explícitamente las condiciones y eventos que provocan un cambio de estado.
            \item Modelar lógicas de excepción complejas (pagos fallidos, falta de stock).
            \item Simular el proceso bajo diferentes escenarios para identificar cuellos de botella.
            \item Crear un modelo lógico que sirva como especificación para un sistema real.
        \end{itemize}
    \end{alertblock}
\end{frame}

%------------------------------------------------
\subsection{Fase 1: Modelo Básico del Ciclo de Vida de la Orden}
%------------------------------------------------

\begin{frame}{Fase 1: El 'Camino Feliz'}

    \begin{block}{Objetivo}
    Nos centraremos en el 'camino feliz' (\textit{happy path}): un flujo de proceso lineal donde no ocurren excepciones. El objetivo es familiarizarnos con la creación de estados, transiciones y la interacción básica entre Simulink y Stateflow.
    \end{block}
    
    \begin{alertblock}{Estados del Proceso Básico}
        \begin{itemize}
            \item \texttt{Inactivo} $\rightarrow$ \texttt{Recibida} $\rightarrow$ \texttt{Pago\_Verificado} $\rightarrow$ \texttt{Inventario\_Asignado} $\rightarrow$ \texttt{Orden\_Enviada} $\rightarrow$ \texttt{Orden\_Entregada}
        \end{itemize}
    \end{alertblock}
\end{frame}

%------------------------------------------------

\begin{frame}[fragile]{Paso 1: Crear el Tipo de Dato Enumerado}
    \begin{block}{¿Por qué usar una enumeración?}
    Para mejorar la legibilidad y robustez del modelo, usamos un tipo de dato enumerado para representar los estados. Esto nos permite usar nombres descriptivos (como \texttt{Recibida}) en lugar de números (1, 2, 3).
    \end{block}

   
\end{frame}

\begin{frame}[fragile]{Paso 1: Crear el Tipo de Dato Enumerado}

 \begin{alertblock}{Código: \texttt{OrdenStatus.m}}
    Cree un archivo \texttt{.m} con el mismo nombre que la clase y guarde este código:
    \begin{lstlisting}
classdef OrdenStatus < Simulink.IntEnumType
  % Enumeracion para los estados de una orden.
  enumeration
      Inactivo(0)
      Recibida(1)
      Pago_Verificado(2)
      Inventario_Asignado(3)
      Orden_Enviada(4)
      Orden_Entregada(5)
  end
end
    \end{lstlisting}
    \end{alertblock}
\end{frame}
%------------------------------------------------

\begin{frame}{Paso 2 y 3: Configurar Simulink y las Variables de Stateflow}
    \begin{columns}[t]
        \column{0.5\textwidth}
            \begin{block}{Modelo en Simulink}
            Se construye la estructura principal que se comunicará con el \textit{Chart} de Stateflow.
                \begin{itemize}
                    \item 5 bloques \texttt{Constant} para simular las entradas de control.
                    \item 1 bloque \texttt{Stateflow Chart} para la lógica.
                    \item 1 bloque \texttt{Display} para ver el estado actual.
                \end{itemize}
            \end{block}

        \column{0.5\textwidth}
            \begin{alertblock}{Variables (Símbolos) en Stateflow}
            Se define la interfaz de datos del \textit{chart}.
                \begin{itemize}
                    \item 5 variables de \textbf{entrada} (booleanas) para las condiciones: \texttt{nueva\_orden}, \texttt{pago\_ok}, etc.
                    \item 1 variable de \textbf{salida} (del tipo \texttt{OrdenStatus}) para reportar el estado: \texttt{estado\_orden}.
                \end{itemize}
            \end{alertblock}
            
    \end{columns}
\end{frame}


\begin{frame}{Paso 2 y 3: Configurar Simulink y las Variables de Stateflow}

\begin{figure}
                \includegraphics[width=0.8\linewidth]{Figuras/Cap10/fig10.2.png}
                \caption{Panel de Símbolos con las variables.}
                \label{fig:sf_fase1_symbols_fig}
            \end{figure}
\end{frame}
%------------------------------------------------

\begin{frame}{Paso 4: Construir el Diagrama de Estados}
    \begin{block}{Lógica del 'Camino Feliz'}
    Se dibujan los estados y se conectan con transiciones.
        \begin{itemize}
            \item Cada estado tiene una acción de entrada (\texttt{entry}) que actualiza la variable de salida \texttt{estado\_orden}.
            \item Cada transición está 'protegida' por una condición que debe ser verdadera para que el sistema avance al siguiente estado (e.g., \texttt{[pago\_ok == 1]}).
        \end{itemize}
    \end{block}
    
\end{frame}


\begin{frame}{Paso 4: Construir el Diagrama de Estados}

\begin{figure}
        \centering
        \includegraphics[width=0.7\textwidth]{Figuras/Cap10/fig10.3.png}
        \caption{Diagrama de Stateflow completo para el modelo básico.}
        \label{fig:sf_fase1_chart}
    \end{figure}
\end{frame}
%------------------------------------------------

\begin{frame}{Paso 5: Simular y Verificar el Modelo}
    \begin{block}{Prueba Interactiva}
        \begin{enumerate}
            \item Configure el tiempo de simulación en \texttt{inf} e inicie la simulación. El \texttt{Display} mostrará \texttt{0} (Inactivo).
            \item Durante la simulación, haga doble clic en el bloque \texttt{Constant} \texttt{nueva\_orden} y cambie su valor a \texttt{1}.
            \item Observe cómo el \textit{chart} se anima, la transición se activa y el \texttt{Display} cambia a \texttt{1} (Recibida).
            \item Repita el proceso para las demás entradas en secuencia para hacer que la orden avance por todos los estados hasta ser entregada.
        \end{enumerate}
    \end{block}
\end{frame}

%------------------------------------------------
\subsection{Fase 2: Introducción de Excepciones y Decisiones}
%------------------------------------------------

\begin{frame}{Fase 2: Manejo de Excepciones y añadir realismo}
    \begin{block}{Objetivo}
    El 'camino feliz' es poco realista. Haremos nuestro modelo más robusto al introducir dos excepciones comunes: un \textbf{fallo en el pago} y la \textbf{falta de inventario}.
    \end{block}
    
    \begin{alertblock}{Herramienta Clave: Conector de Unión (Junction)}
    Utilizaremos el conector de unión, un pequeño círculo que nos permite implementar lógica de decisión tipo \textbf{if-then-else} de manera gráfica para crear bifurcaciones en el flujo del proceso.
    \end{alertblock}
\end{frame}

%------------------------------------------------

\begin{frame}[fragile]{Paso 1: Actualizar el Tipo de Dato Enumerado}
    \begin{block}{Nuevos Estados de Excepción}
    Debemos añadir los nuevos estados a nuestra clase \texttt{OrdenStatus.m} para poder referenciarlos en el modelo.
    \end{block}

    
\end{frame}

\begin{frame}[fragile]{Paso 1: Actualizar el Tipo de Dato Enumerado}

\begin{alertblock}{Código Actualizado: \texttt{OrdenStatus.m}}
    \begin{lstlisting}
classdef OrdenStatus < Simulink.IntEnumType
  enumeration
      Inactivo(0)
      Recibida(1)
      Pago_Verificado(2)
      Inventario_Asignado(3)
      Orden_Enviada(4)
      Orden_Entregada(5)
      % --- Nuevos estados de excepcion ---
      Pago_Fallido(6)
      Fuera_de_Stock(7)
  end
end
    \end{lstlisting}
    \end{alertblock}
\end{frame}
%------------------------------------------------

\begin{frame}{Paso 2: Modificar el Diagrama de Estados con Uniones}
            \begin{block}{Lógica de Decisión}
                \begin{itemize}
                    \item Se añaden los estados \texttt{Pago\_Fallido} y \texttt{Fuera\_de\_Stock}.
                    \item Se elimina la transición directa entre \texttt{Recibida} y \texttt{Pago\_Verificado}.
                    \item Se inserta un \textbf{conector de unión}.
                    \item Desde la unión, una transición va a \texttt{Pago\_Verificado} con la condición \texttt{[pago\_ok == 1]}.
                    \item Otra transición va a \texttt{Pago\_Fallido} con la condición \texttt{[pago\_ok == 0]}.
                    \item Se repite un proceso similar para la decisión del inventario.
                \end{itemize}
            \end{block}
\end{frame}

\begin{frame}{Paso 2: Modificar el Diagrama de Estados con Uniones}

\begin{figure}
                \centering
                \includegraphics[width=0.8\linewidth]{Figuras/Cap10/fig10.4.png}
                \caption{Diagrama con lógica de decisión para manejar excepciones.}
                \label{fig:sf_fase2_chart}
            \end{figure}
\end{frame}
%------------------------------------------------

\begin{frame}{Paso 3: Simular los Escenarios de Excepción}
    \begin{block}{Prueba de Fallo de Pago}
        \begin{enumerate}
            \item Activar \texttt{nueva\_orden = 1}. El sistema transita a \texttt{Recibida}.
            \item Asegurarse de que \texttt{pago\_ok = 0}.
            \item En el siguiente paso, el sistema tomará el camino hacia \texttt{Pago\_Fallido}.
            \item Después de 5 segundos, la orden se cancela y el sistema vuelve a \texttt{Inactivo}.
        \end{enumerate}
    \end{block}
    
    \begin{alertblock}{Prueba de Falta de Stock}
        \begin{enumerate}
            \item Activar \texttt{nueva\_orden = 1} y \texttt{pago\_ok = 1}. El sistema llega a \texttt{Pago\_Verificado}.
            \item Asegurarse de que \texttt{stock\_disponible = 0}.
            \item El sistema tomará la ruta hacia \texttt{Fuera\_de\_Stock} y luego volverá a \texttt{Inactivo}.
        \end{enumerate}
    \end{alertblock}
\end{frame}

%------------------------------------------------
\subsection{Fase 3: Modelado de Procesos Paralelos y Jerarquía}
%------------------------------------------------

\begin{frame}{Fase 3: Añadiendo Complejidad con Jerarquía y Paralelismo}

    \begin{block}{Objetivo}
    Los procesos logísticos reales rara vez son secuenciales. Usaremos dos conceptos avanzados de Stateflow para modelar subprocesos que ocurren de forma concurrente:
        \begin{itemize}
            \item \textbf{Jerarquía (Superestados):} Para agrupar la lógica principal del procesamiento de la orden.
            \item \textbf{Paralelismo (Estados Ortogonales):} Para modelar la cancelación de una orden o el proceso de devolución, que pueden ocurrir en paralelo al flujo principal.
        \end{itemize}
    \end{block}
\end{frame}

%------------------------------------------------

\begin{frame}[fragile]{Paso 1: Actualizar Enumeración y Variables}
    \begin{columns}[t]
        \column{0.5\textwidth}
            \begin{alertblock}{Nuevos Estados}
            Añadimos estados para la cancelación y el ciclo de devolución a \texttt{OrdenStatus.m}.
            \begin{lstlisting}
    % ... (estados anteriores)
    Cancelada(8)
    Devolucion_Solicitada(9)
    Paquete_Recibido_Dev(10)
    Reembolso_Procesado(11)
            \end{lstlisting}
            \end{alertblock}
        \column{0.5\textwidth}
            \begin{block}{Nuevas Variables}
            Añadimos nuevas variables de \textbf{entrada} y \textbf{salida} al \textit{chart} para gestionar los nuevos procesos.
                \begin{itemize}
                    \item \texttt{solicitud\_cancelar}
                    \item \texttt{solicitud\_devolver}
                    \item \texttt{paquete\_devuelto}
                    \item \texttt{estado\_devolucion}
                \end{itemize}
            \end{block}
    \end{columns}
\end{frame}

%------------------------------------------------

\begin{frame}{Paso 2: Reestructurar con Jerarquía y Paralelismo}
    \begin{columns}[t]
        \column{0.5\textwidth}
            \begin{block}{Crear un Superestado}
                \begin{itemize}
                    \item Se crea un estado grande \texttt{Orden\_Activa} y se arrastran todos los estados del proceso principal (excepto \texttt{Inactivo}) dentro de él.
                    \item Una única transición desde el \textbf{borde} de este superestado hacia un nuevo estado \texttt{Cancelada} permite interrumpir \textit{cualquier} subestado activo con la condición \texttt{[solicitud\_cancelar == 1]}.
                \end{itemize}
            \end{block}
        \column{0.5\textwidth}
            \begin{alertblock}{Añadir Paralelismo}
                \begin{itemize}
                    \item El superestado \texttt{Orden\_Activa} se descompone en dos regiones paralelas (AND) con una línea discontinua.
                    \item Una región mantiene el flujo principal del pedido.
                    \item La otra región contiene una nueva máquina de estados para gestionar el proceso de \textbf{devolución}, que se activa solo cuando el proceso principal llega al estado \texttt{Orden\_Entregada}.
                \end{itemize}
            \end{alertblock}
    \end{columns}
\end{frame}

%------------------------------------------------

\begin{frame}{Diagrama Final del Modelo}
    \begin{figure}
        \centering
        \includegraphics[width=0.8\textwidth]{Figuras/Cap10/fig10.5.png}
        \caption{Diagrama de Stateflow final con jerarquía y estados paralelos.}
        \label{fig:sf_fase3_chart}
    \end{figure}
\end{frame}

\begin{frame}{Diagrama Final del Modelo}
    \begin{figure}
        \centering
        \includegraphics[width=0.9\textwidth]{Figuras/Cap10/fig10.6.png}
        \caption{Diagrama de Stateflow final con jerarquía y estados paralelos.}
        \label{fig:sf_fase3_chart}
    \end{figure}
\end{frame}

%------------------------------------------------

\begin{frame}{Paso 3: Simular Escenarios Avanzados}
    \begin{columns}[t]
        \column{.48\textwidth}
            \begin{block}{Simular una Cancelación}
                \begin{enumerate}
                    \item Iniciar el proceso de una orden hasta un estado intermedio (e.g., \texttt{Inventario\_Asignado}).
                    \item Activar la señal \texttt{solicitud\_cancelar = 1}.
                    \item Observar cómo la ejecución salta inmediatamente del subestado activo al estado \texttt{Cancelada}, demostrando el poder de la transición de alto nivel.
                \end{enumerate}
            \end{block}

        \column{.48\textwidth}
            \begin{alertblock}{Simular una Devolución}
                 \begin{enumerate}
                    \item Completar el flujo principal hasta que la orden llegue a \texttt{Orden\_Entregada}.
                    \item En este punto, la máquina de estados paralela de devolución se activa.
                    \item Activar \texttt{solicitud\_devolver = 1}.
                    \item Observar cómo el subproceso de devolución avanza por sus propios estados, mientras el estado principal de la orden permanece en \texttt{Orden\_Entregada}.
                \end{enumerate}
            \end{alertblock}
    \end{columns}
\end{frame}
\end{document}