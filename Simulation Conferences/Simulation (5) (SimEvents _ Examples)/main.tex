\documentclass{beamer}
\usepackage{ragged2e}
\usepackage{times}
\usepackage{array}
\usepackage{amsmath}
\usepackage{amssymb}
\usepackage{amsfonts}
\usepackage{mathtools}
\usepackage{booktabs}
\newcolumntype{L}[1]{>{\raggedright\arraybackslash}p{#1}}
\usepackage[spanish]{babel}
\usepackage{caption}
\usepackage{setspace}
\usepackage{changepage}
\usepackage{hyperref}
\usepackage{listings}
\usepackage{tikz}
\usetikzlibrary{shapes.geometric, arrows.meta, positioning}
\tikzstyle{block} = [rectangle, draw, text centered, minimum height=2em]
\tikzstyle{arrow} = [thick,->,>=stealth]
\justifying

\usetheme{cambridge}
\usefonttheme{professionalfonts}

\setbeamertemplate{footline}{%
  \hbox{%
    \begin{beamercolorbox}[wd=0.8\paperwidth,ht=2.25ex,dp=1ex,leftskip=1em]{author in head/foot}%
      \usebeamerfont{author in head/foot}BARSEKH-ONJI Aboud, ORCID: 0009-0004-5440-8092
    \end{beamercolorbox}%
    \begin{beamercolorbox}[wd=0.2\paperwidth,ht=2.25ex,dp=1ex,rightskip=1em]{page number in head/foot}%
      \usebeamerfont{page number in head/foot}\insertframenumber{} / \inserttotalframenumber
    \end{beamercolorbox}%
  }%
}

\institute{\textbf{Universidad Anáhuac México, Facultad de Ingeniería}}
\title{\textbf{Discrete Events Simulation (Examples)}}
\author{\textbf{BARSEKH-ONJI Aboud (DSc.)}}
\date{\today}

\begin{document}

\begin{frame}
    \titlepage
\end{frame}

\begin{frame}
    \frametitle{Tabla de Contenidos}
    \tableofcontents
\end{frame}

\section{Ejemplo 1}
\subsection{Descripción del Sistema}
\begin{frame}{Sistema de Atención al Cliente en un Banco}
Este modelo simula un banco donde los clientes llegan, esperan en una cola, son atendidos por un cajero y luego salen del sistema. Usa bloques de \textbf{SimEvents} y una función de \textbf{Simulink} para registrar los tiempos de espera de los clientes.
\end{frame}
\subsection{Diagrama del Modelo}
\begin{frame}{Diagrama del Modelo de Atención Bancaria}
\justifying
\vspace{-0.3cm}
Este diagrama representa el flujo de entidades en el sistema bancario simulado. Los bloques corresponden a componentes de \textbf{SimEvents} conectados para modelar el proceso desde la llegada hasta la salida del cliente.

\vspace{0.5cm}
\centering
\begin{tikzpicture}[node distance=0.8cm and 0.8cm, auto]
    \node[block] (gen) {\footnotesize Entity Generator};
    \node[block, right=of gen] (queue) {\footnotesize Entity Queue};
    \node[block, right=of queue] (server) {\footnotesize Entity Server};
    \node[block, right=of server] (term) {\footnotesize Entity Terminator};
    \node[block, below=of server] (func) {\footnotesize Simulink Function};

    \draw[arrow] (gen) -- (queue);
    \draw[arrow] (queue) -- (server);
    \draw[arrow] (server) -- (term);
    \draw[arrow] (server) -- (func);
\end{tikzpicture}
\end{frame}

\subsection{Componentes del Modelo}
\begin{frame}{Componentes del Modelo}
\begin{itemize}
    \item \textbf{Entity Generator:} Genera clientes con interarribo aleatorio.
    \item \textbf{Entity Queue:} Cola de espera con disciplina FIFO.
    \item \textbf{Entity Server:} Cajero con tiempo de servicio aleatorio.
    \item \textbf{Entity Terminator:} Clientes salen del sistema tras ser atendidos.
    \item \textbf{Simulink Function:} Registra tiempo de espera por cliente.
\end{itemize}
\end{frame}

\begin{frame}{Entity Generator}
\begin{itemize}
    \item Fuente de tiempo: \texttt{MATLAB action}
    \item Código para tiempo entre llegadas:
\end{itemize}
\begin{block}{Código}
\texttt{dt = rand(1,1);}
\end{block}
\end{frame}

\begin{frame}{Entity Queue y Server}
\textbf{Entity Queue:}
\begin{itemize}
    \item Capacidad: ilimitada o fija.
    \item Disciplina: FIFO.
\end{itemize}
\vspace{0.3cm}
\textbf{Entity Server:}
\begin{itemize}
    \item Tiempo de servicio: \texttt{exprnd(1)}
    \item Acción de salida: \texttt{logTime();}
\end{itemize}
\end{frame}

\begin{frame}{Simulink Function: logWaitTime}
Función llamada desde el servidor tras atender al cliente. Registra el tiempo de espera en una variable del workspace.
\end{frame}




\section{Ejemplo 2}
\subsection{Descripción del Sistema}
\begin{frame}{Descripción del Sistema}
\justifying
Este modelo simula el proceso de inflado de llantas en una línea de producción. Cada llanta se infla inicialmente y luego se inspecciona su presión. Si la presión está dentro del rango aceptable (30-34 psi), se aprueba. Si no, se envía a una estación de ajuste para corregir la presión.
\end{frame}

\subsection{Bloques Principales}
\begin{frame}{Componentes del Modelo}
\begin{itemize}
    \item \textbf{Entity Generator}: Genera llantas con presión inicial aleatoria (28–36 psi).
    \item \textbf{Entity Server (Inflado)}: Simula el proceso inicial de inflado.
    \item \textbf{Entity Server (Inspección)}: Revisa si la presión está en rango.
    \item \textbf{Simulink Function}: Evalúa si la llanta aprueba.
    \item \textbf{Switch}: Direcciona según atributo `Aprobada`.
    \item \textbf{Entity Server (Ajuste)}: Corrige presión fuera de rango.
    \item \textbf{Entity Terminator}: Sale del sistema.
\end{itemize}
\end{frame}

\subsection{Configuración de Bloques}
\begin{frame}[fragile]{Entity Generator}
\justifying
\textbf{Tiempo entre llegadas:} Exponencial con media de 5 s.\\
\textbf{Acciones de creación:}
\begin{block}{Código}
\texttt{entity.Presion = 28 + (36-28)*rand();}
\end{block}
\end{frame}

\begin{frame}[fragile]{Entity Server (Inflado)}
\textbf{Tiempo de servicio:} 10 segundos\\
\textbf{Acciones de salida:}
\begin{block}{Código}
\texttt{entity.Presion = entity.Presion + 2 + (4-2)*rand();}
\end{block}
\end{frame}

\begin{frame}[fragile]{Entity Server (Inspección)}
\textbf{Tiempo de servicio:} 5 segundos\\
\textbf{Acciones de salida:}
\begin{block}{Código}
\texttt{entity.Aprobada = EvaluarPresion(entity.Presion);}
\end{block}
\end{frame}

\begin{frame}[fragile]{Simulink Function: EvaluarPresion}
\begin{block}{Código}
\begin{verbatim}
function aprobada = EvaluarPresion(presion)
    if presion >= 30 && presion <= 34
        aprobada = 1;
    else
        aprobada = 2;
    end
end
\end{verbatim}
\end{block}
\end{frame}

\begin{frame}[fragile]{Entity Server (Ajuste de Presión)}
\textbf{Tiempo de servicio:} 8 segundos\\
\textbf{Acciones de salida:}
\begin{block}{Código}
\texttt{entity.Presion = 30 + (34-30)*rand();}
\end{block}
\end{frame}



\subsection{Notas Finales}
\begin{frame}{Notas y Expansiones Posibles}
\begin{itemize}
\justifying
    \item Las llantas tienen atributos personalizados (`Presion`, `Aprobada`).
    \item Se pueden analizar métricas como tiempos promedio y tasa de rechazo.
    \item Se puede extender a varios infladores, inspecciones paralelas o estadísticas.
\end{itemize}
\end{frame}

\end{document}
