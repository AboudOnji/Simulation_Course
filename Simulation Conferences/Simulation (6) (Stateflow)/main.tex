
\documentclass{beamer}
\usepackage{ragged2e}
\usepackage{times}
\usepackage{array}
\usepackage{amsmath}
\usepackage{amssymb}
\usepackage{amsfonts}
\usepackage{mathtools}
\usepackage{booktabs}
\newcolumntype{L}[1]{>{\raggedright\arraybackslash}p{#1}}
\usepackage[spanish]{babel}
\usepackage{caption}
\usepackage{setspace}
\usepackage{changepage}
\usepackage{hyperref}
\usepackage{listings}
\usepackage{tikz}
\usetikzlibrary{shapes.geometric, arrows.meta, positioning}
\tikzstyle{block} = [rectangle, draw, text centered, minimum height=2em]
\tikzstyle{arrow} = [thick,->,>=stealth]
\justifying

\usetheme{cambridge}
\usefonttheme{professionalfonts}

\setbeamertemplate{footline}{%
  \hbox{%
    \begin{beamercolorbox}[wd=0.8\paperwidth,ht=2.25ex,dp=1ex,leftskip=1em]{author in head/foot}%
      \usebeamerfont{author in head/foot}BARSEKH-ONJI Aboud, ORCID: 0009-0004-5440-8092
    \end{beamercolorbox}%
    \begin{beamercolorbox}[wd=0.2\paperwidth,ht=2.25ex,dp=1ex,rightskip=1em]{page number in head/foot}%
      \usebeamerfont{page number in head/foot}\insertframenumber{} / \inserttotalframenumber
    \end{beamercolorbox}%
  }%
}

\institute{\textbf{Universidad Anáhuac México, Facultad de Ingeniería}}
\title{\textbf{Comparación de Algoritmos de Optimización: PSO vs GA}}
\author{\textbf{BARSEKH-ONJI Aboud (DSc.)}}
\date{\today}

\begin{document}

\begin{frame}
    \titlepage
\end{frame}

\begin{frame}
    \frametitle{Tabla de Contenidos}
    \tableofcontents
\end{frame}

\section{Introducción}
\begin{frame}{Introducción}
\begin{itemize}
    \item \textbf{Objetivo}: Comparar la convergencia de los algoritmos de Optimización por Enjambre de Partículas (PSO) y Algoritmos Genéticos (GA) en la optimización de la función de Rastrigin.
    \item \textbf{Función de Rastrigin}: $$f(x) = 10n + \sum_{i=1}^{n} \left[ x_i^2 - 10 \cos(2 \pi x_i) \right]$$
    \item \textbf{Características}: Múltiples mínimos locales, ideal para evaluar algoritmos de optimización.
\end{itemize}
\end{frame}

\section{Teoría de PSO}
\begin{frame}{Teoría de PSO}
\begin{itemize}
    \item \textbf{Inspiración}: Comportamiento colectivo de aves y peces.
    \item \textbf{Mecanismo}:
    \begin{itemize}
        \item \textbf{Partículas}: Representan posibles soluciones.
        \item \textbf{Movimiento}: Ajuste basado en la experiencia propia y la de los vecinos.
    \end{itemize}
    \item \textbf{Ventajas}:
    \begin{itemize}
        \item Simplicidad y rapidez.
        \item Adaptabilidad a problemas no lineales y multiobjetivo.
    \end{itemize}
    \item \textbf{Desventajas}:
    \begin{itemize}
        \item Tendencia a quedar atrapado en óptimos locales.
    \end{itemize}
\end{itemize}
\end{frame}

\section{Teoría de GA}
\begin{frame}{Teoría de GA}
\begin{itemize}
    \item \textbf{Inspiración}: Evolución biológica.
    \item \textbf{Mecanismo}:
    \begin{itemize}
        \item \textbf{Individuos}: Representan soluciones mediante cromosomas.
        \item \textbf{Operadores}: Mutación, cruce y selección.
    \end{itemize}
    \item \textbf{Ventajas}:
    \begin{itemize}
        \item Exploración de grandes espacios de búsqueda.
        \item Robustez ante problemas complejos y no lineales.
    \end{itemize}
    \item \textbf{Desventajas}:
    \begin{itemize}
        \item Requiere más tiempo computacional.
    \end{itemize}
\end{itemize}
\end{frame}

\section{Implementación en MATLAB}
\begin{frame}[fragile]{Implementación en MATLAB}
\begin{itemize}
    \item \textbf{Función de Rastrigin}:
    \begin{block}{Código}
    \texttt{rastrigin = @(x) 10 * numel(x) + sum(x.\^2 - 10 * cos(2 * pi * x));}
    \end{block}
    \item \textbf{Parámetros comunes}:
    \begin{itemize}
        \item Dimensión del problema: \texttt{dim = 10;}
        \item Número máximo de iteraciones: \texttt{max_iter = 100;}
        \item Límite inferior: \texttt{lb = -5.12;}
        \item Límite superior: \texttt{ub = 5.12;}
    \end{itemize}
\end{itemize}
\end{frame}

\begin{frame}[fragile]{PSO}
\begin{itemize}
    \item \textbf{Configuración}:
    \begin{block}{Código}
    \texttt{options\_pso = optimoptions('particleswarm', 'SwarmSize', 30, 'MaxIterations', max\_iter, 'Display', 'off');}
    \end{block}
    \item \textbf{Ejecución}:
    \begin{block}{Código}
    \texttt{[x\_pso, fval\_pso, exitflag\_pso, output\_pso] = particleswarm(rastrigin, dim, lb, ub, options\_pso);}
    \end{block}
\end{itemize}
\end{frame}

\begin{frame}[fragile]{GA}
\begin{itemize}
    \item \textbf{Configuración}:
    \begin{block}{Código}
    \texttt{options\_ga = optimoptions('ga', 'PopulationSize', 30, 'MaxGenerations', max\_iter, 'Display', 'off');}
    \end{block}
    \item \textbf{Ejecución}:
    \begin{block}{Código}
    \texttt{[x\_ga, fval\_ga, exitflag\_ga, output\_ga] = ga(rastrigin, dim, [], [], [], [], lb, ub, [], options\_ga);}
    \end{block}
\end{itemize}
\end{frame}

\begin{frame}[fragile]{Resultados}
\begin{itemize}
    \item \textbf{PSO}:
    \begin{block}{Código}
    \texttt{fprintf('PSO: Mejor valor encontrado = %.4f en %d iteraciones\n', fval\_pso, output\_pso.iterations);}
    \end{block}
    \item \textbf{GA}:
    \begin{block}{Código}
    \texttt{fprintf('GA: Mejor valor encontrado = %.4f en %d generaciones\n', fval\_ga, output\_ga.generations);}
    \end{block}
\end{itemize}
\end{frame}

\begin{frame}[fragile]{Gráfica de Convergencia}
\begin{itemize}
    \item \textbf{Código para la gráfica}:
    \begin{block}{Código}
    \begin{verbatim}
figure;
plot(output_pso.bestfvals, 'b', 'LineWidth', 2);
hold on;
plot(output_ga.bestfvals, 'r', 'LineWidth', 2);
xlabel('Iteraciones/Generaciones');
ylabel('Mejor valor de la función');
legend('PSO', 'GA');
title('Comparación de convergencia: PSO vs GA');
grid on;
    \end{verbatim}
    \end{block}
\end{itemize}
\end{frame}

\section{Análisis de Convergencia}
\begin{frame}{Análisis de Convergencia}
\begin{itemize}
    \item \textbf{PSO}:
    \begin{itemize}
        \item Convergencia rápida.
        \item Posible estancamiento en óptimos locales.
    \end{itemize}
\end{itemize}
\end{frame}
\end{document}