\documentclass{beamer}
\usetheme{Berkeley}
\usepackage[utf8]{inputenc}
\usepackage[spanish]{babel}
\usepackage{graphicx}
\usepackage{amsmath}

\title{Definir el comportamiento del gráfico usando acciones de estado y etiquetas de transición}
\author{Prof. DSc. BARSEKH-ONJI Aboud}
\institute{Universidad Anáhuac México\\ Facultad de Ingeniería}
\date \today

\begin{document}

\frame{\titlepage}
\begin{frame}{Contenido}
    \tableofcontents
\end{frame}
%---------------------------------------
\section{Objetivo}
\begin{frame}{Objetivo}
    Comprender cómo se puede definir el comportamiento de un gráfico de Stateflow mediante:
    \begin{itemize}
        \item Acciones de estado (\textit{entry}, \textit{during}, \textit{exit})
        \item Etiquetas de transición con condiciones y acciones
    \end{itemize}
    \vspace{0.5cm}
    Caso de estudio: Verificación empírica de la conjetura de Collatz.
\end{frame}

%---------------------------------------
\section{Caso: Regla de la conjetura de Collatz}
\begin{frame}{Regla de la conjetura de Collatz}
    Dado un número entero positivo $n$:
    \begin{itemize}
        \item Si $n_i$ es par, entonces $n_{i+1} = n_i / 2$
        \item Si $n$ es impar, entonces $n_{i+1} = 3n_i + 1$
    \end{itemize}
    El proceso se repite hasta llegar a $n = 1$
\end{frame}

%--------------------------------------
\section{Estructura del gráfico}
\begin{frame}{Estructura del gráfico}
    \begin{itemize}
        \item 3 estados: \textbf{Init}, \textbf{Even}, \textbf{Odd}
        \item Transición predeterminada inicial hacia Init
        \item Transiciones condicionales entre Even y Odd
    \end{itemize}
    \vspace{0.3cm}
    \centering\includegraphics[width=0.6\linewidth]{StateTransitionActionsGetStartedExample_01.png}
\end{frame}

\section{Tipos de acciones de estado}
\begin{frame}{Tipos de acciones de estado}
    \textbf{Entry:} cuando el estado se vuelve activo \\
    \textbf{During:} en cada instante mientras el estado está activo \\
    \textbf{Exit:} cuando el estado se vuelve inactivo
    \vspace{0.4cm}
    \begin{block}{Ejemplo combinado}
        \texttt{entry, during:} se ejecuta al entrar y mientras el estado permanece activo
    \end{block}
\end{frame}

\begin{frame}{Tipos de acciones de estado}
    \centering\includegraphics[width=0.9\linewidth]{StateTransitionActionsGetStartedExample_01.png}
\end{frame}
%---------------------------------------

\begin{frame}{Acciones del estado}
   
    \begin{block}{Init}
        cuando este estado se convierte en activo al inicio de la simulación, la acción entry determina la paridad de n y establece y en false. Cuando el gráfico sale de Init después de una unidad de tiempo, la acción exit determina si n es igual a uno.
    \end{block}

\end{frame}

\begin{frame}{Acciones del estado}
   
   \begin{block}{Even}
       cuando este estado se convierte en activo y en cada unidad de tiempo posterior en la que el estado está activo, la acción combinada entry, during calcula el valor y la paridad para el siguiente número de la secuencia de granizo, n/2.
    \end{block}
\end{frame}

\begin{frame}{Acciones del estado}
   
    \begin{block}{Odd}
        cuando este estado se convierte en activo y en cada unidad de tiempo posterior en la que el estado está activo, la acción combinada entry, during comprueba si n es mayor que uno y, si lo es, calcula el valor y la paridad para el siguiente número de la secuencia de granizo, 3*n+1.
    \end{block}

\end{frame}
 
\section{Configuración detallada del modelo}
\begin{frame}{Acciones en Init}
    \textbf{Entry:}
    \begin{itemize}
        \item $n2 = rem(n,2)$
        \item $y = \text{false}$
    \end{itemize}
    \textbf{Exit:}
    \begin{itemize}
        \item $y = isequal(n,1)$
    \end{itemize}
    \textbf{Transición desde Init:}
    \begin{itemize}
        \item Si $n2 == 0$: transición a Even
        \item Si $n2 \neq 0$: transición a Odd
    \end{itemize}
\end{frame}

%---------------------------------------


\begin{frame}{Acciones en Even}
    \textbf{Entry, During:}
    \begin{itemize}
        \item $n = n / 2$
        \item $n2 = rem(n,2)$ 
    \end{itemize}
    \textbf{Transición:}
    \begin{itemize}
        \item Si $n2 \neq 0$: transición a Odd
        \item Acción de transición: $y = isequal(n,1)$
    \end{itemize}
\end{frame}

%---------------------------------------
\begin{frame}{Acciones en Odd}
    \textbf{Entry, During:}
    \begin{itemize}
        \item Si $n > 1$:
        \begin{itemize}
            \item $n = 3n + 1$
            \item $n2 = rem(n,2)$
        \end{itemize}
    \end{itemize}
    \textbf{Transición:}
    \begin{itemize}
        \item Si $n2 == 0$: transición a Even
    \end{itemize}
\end{frame}

%---------------------------------------
\begin{frame}{Simulación: valor inicial $n = 9$}
    \begin{itemize}
        \item Transición a Init: $n = 9$, $n2 = 1$
        \item Init $\rightarrow$ Odd: $n = 28$
        \item Odd $\rightarrow$ Even: $n = 14$
        \item Even $\rightarrow$ Odd: $n = 22$
        \item $\cdots$
        \item Proceso termina cuando $n = 1$
    \end{itemize}
    \vspace{0.3cm}
    \centering \includegraphics[width=0.7\linewidth]{xxsf_collatz-sdi.png}
\end{frame}

%---------------------------------------

%---------------------------------------
\section{Etiquetas de transición}
\begin{frame}{Etiquetas de transición}
    \textbf{Formato general:} \texttt{[Condición]\{Acción de condición\}}
    \vspace{0.3cm}
    \begin{itemize}
        \item La condición determina si se realiza la transición
        \item La acción de condición se ejecuta antes de salir del estado origen
    \end{itemize}
    \textbf{Ejemplo:} \texttt{[n2 $\sim=$ 0]\{y = isequal(n,1)\}}
\end{frame}

%---------------------------------------
\section{Resumen}
\begin{frame}{Resumen}
    \begin{itemize}
        \item Las acciones de estado permiten definir la lógica dentro de cada estado
        \item Las etiquetas de transición controlan el flujo entre estados
        \item El ejemplo de la secuencia de Collatz muestra una aplicación empírica clara
        \item El gráfico termina la simulación automáticamente cuando $n = 1$
    \end{itemize}
\end{frame}

\end{document}
