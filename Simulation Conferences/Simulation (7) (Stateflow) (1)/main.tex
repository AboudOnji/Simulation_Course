\documentclass{beamer}
\usetheme{Berkeley}
\usepackage[utf8]{inputenc}
\usepackage[spanish]{babel}
\usepackage{graphicx}
\usepackage{amsmath}

\title{Introducción a Stateflow}
\author{Prof. DSc. BARSEKH-ONJI Aboud}
\institute{Universidad Anáhuac México\\ Facultad de Ingeniería}
\date{\today}

\begin{document}

\frame{\titlepage}

% Índice
\begin{frame}{Contenido}
    \tableofcontents
\end{frame}

%---------------------------------------
\section{¿Qué es Stateflow?}
\begin{frame}{¿Qué es Stateflow?}
    Stateflow es una herramienta de MATLAB/Simulink para modelar:
    \begin{itemize}
        \item Máquinas de estado finito
        \item Lógica de control basada en eventos
        \item Diagramas de flujo y decisiones condicionales
    \end{itemize}
    Permite representar visualmente el comportamiento lógico de sistemas reactivos.
\end{frame}

%---------------------------------------
\section{Ejemplo 1: Rectificador de media onda}
\begin{frame}{Ejemplo 1: Rectificador}
    Este gráfico implementa la lógica de un rectificador de media onda:
    \begin{itemize}
        \item Estado \textbf{On}: salida $y = x$
        \item Estado \textbf{Off}: salida $y = 0$
        \item Transiciones según el umbral $t_0$
    \end{itemize}
    \vspace{0.4cm}
    \centering\includegraphics[width=0.5\linewidth]{rectify-step4.png}
\end{frame}

\begin{frame}{Simulación del modelo}
    \centering\includegraphics[width=0.7\linewidth]{rectify-scope.png}
    
\end{frame}

\begin{frame}{Simulación del modelo}
    \centering\includegraphics[width=0.4\linewidth]{rectify-scope.png}
    \vspace{0.4cm}
    \begin{itemize}
        \item En la parte superior se muestra la señal de entrada $x$
        \item En la parte inferior, la señal de salida $y$ con valores negativos filtrados
    \end{itemize}
\end{frame}
%---------------------------------------
\section{Construcción del gráfico}
\begin{frame}{Pasos para crear un gráfico}
    \begin{enumerate}
        \item Crear modelo con `sfnew rectify`
        \item Agregar los estados y las transiciones
        \item Establecer condiciones: $x < t_0$ y $x \geq t_0$
        \item Resolver símbolos: entradas, salidas y constantes
    \end{enumerate}
    \centering\includegraphics[width=0.6\linewidth]{stateflow-editor.png}
\end{frame}

\begin{frame}{Edición del gráfico}
    \begin{itemize}
        \item Estados: On y Off
        \item Transiciones con condiciones en corchetes
        \item Acciones dentro de cada estado
    \end{itemize}
    \vspace{0.3cm}
    \centering\includegraphics[width=0.6\linewidth]{rectify-step2.png}
    \hfill
    
\end{frame}

%---------------------------------------
\section{Resolución de símbolos}
\begin{frame}{Símbolos en Stateflow}
    Antes de simular el modelo, es necesario declarar todos los símbolos:
    \begin{itemize}
        \item $x$: entrada
        \item $t_0$: dato constante
        \item $y$: salida
    \end{itemize}
    \vspace{0.3cm}
    \centering\includegraphics[width=0.85\linewidth]{rectify-resolve-symbols.png}
\end{frame}

%---------------------------------------
\section{Transiciones y estados}
\begin{frame}{Gráficos posibles}
    Ejemplo de un gráfico básico con solo estado On:\\
    \centering\includegraphics[width=0.4\linewidth]{rectify-step3.png}
    \vspace{0.3cm}
    
    \begin{itemize}
        \item Estado \textbf{Off}
        \item Transición con condición $[x<t_0]$
        \item Transición de regreso con $[x\geq t_0]$
    \end{itemize}
\end{frame}

\begin{frame}{Gráfico completo}
    \centering\includegraphics[width=0.55\linewidth]{rectify-step4.png}
    \vspace{0.4cm}
   \\ Este gráfico contiene toda la lógica del rectificador.
\end{frame}

%---------------------------------------
\section{Modelo en Simulink}
\begin{frame}{Modelo completo}
    \centering\includegraphics[width=0.6\linewidth]{rectify-model.png}
    \vspace{0.4cm}
    \begin{itemize}
        \item Fuente: \textbf{Sine Wave}
        \item Gráfico: \textbf{Chart}
        \item Visualización: \textbf{Scope}
    \end{itemize}
\end{frame}

%---------------------------------------
\section{Ejemplo2: Simulación de un semáforo}
\begin{frame}{Ejemplo2: Simulación de un semáforo}
 \centering\includegraphics[width=0.9\linewidth]{semaforo.png}   
\end{frame}

\section{Resumen}
\begin{frame}{Resumen}
    \begin{itemize}
        \item Stateflow permite modelar lógica de control con estados y transiciones.
        \item Los símbolos deben estar bien definidos antes de simular.
        \item El entorno facilita la depuración visual y la simulación paso a paso.
        \item Es útil para representar sistemas reactivos y condicionales.
    \end{itemize}
\end{frame}

\end{document}
