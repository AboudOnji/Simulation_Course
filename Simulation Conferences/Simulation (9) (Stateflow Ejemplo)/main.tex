\documentclass{beamer}
\usetheme{Berkeley}
\usepackage[utf8]{inputenc}
\usepackage[spanish]{babel}
\usepackage{graphicx}
\usepackage{amsmath}
\usecolortheme{dolphin}

%--- Información de la Presentación ---
\title{Modelado de Sistemas Lógicos con Stateflow \\ Ejemplo: Ciclo de Lavadora Automática}
\author{Prof. D.Sc. BARSEKH-ONJI Aboud \inst{1}}
\date{\today}
\institute[Institución]{%
  \inst{1}%
  Facultad de Ingeniería \\ % Ajusta el nombre si es necesario
  Universidad Anáhuac México 
}

%--- Inicio del Documento ---
\begin{document}

%--- Diapositiva de Título ---
\begin{frame}
  \titlepage
\end{frame}

 \begin{frame}{Tabla de Contenido}
  \tableofcontents
\end{frame}
%--- Tabla de Contenido (Opcional) ---
% \begin{frame}{Tabla de Contenido}
%   \tableofcontents
% \end{frame}

%===============================================================================
% Sección 1: Introducción a Stateflow (Repaso)
%===============================================================================
\section{Stateflow: Modelado de Lógica}

\begin{frame}{Recordando Stateflow}
  \begin{itemize}
    \item Stateflow es una herramienta para modelar sistemas lógicos y reactivos usando diagramas de estados finitos.
    \item Componentes clave:
    \begin{itemize}
        \item \textbf{Estados:} Modos de operación (\texttt{Estado}).
        \item \textbf{Transiciones:} Cambios entre estados ($\rightarrow$).
        \item \textbf{Eventos/Condiciones:} Disparadores (\texttt{[condicion]}, \texttt{evento}).
        \item \textbf{Acciones:} Código asociado (\texttt{entry:}, \texttt{during:}, \texttt{exit:}, \texttt{\{accion\}}).
        \item \textbf{Lógica Temporal:} Tiempo (\texttt{after(N, unit)}).
    \end{itemize}
    \item Se integra con Simulink.
  \end{itemize}
\end{frame}

%===============================================================================
% Sección 2: El Ejemplo del Ciclo de Lavadora
%===============================================================================
\section{Nuestro Ejemplo: Ciclo de Lavadora}

\begin{frame}{Sistema a Modelar: Ciclo Básico}
  \begin{itemize}
    \item Modelaremos un ciclo de lavadora simple.
    \item Fases (estados): Inactivo, Lavando, Enjuagando, Centrifugando, Terminado.
    \item La transición depende del tiempo y una señal de inicio.
    \item Queremos saber la fase actual.
  \end{itemize}
\end{frame}

\begin{frame}{Estados del Ciclo de Lavadora}
  Los estados principales son:
  \begin{itemize}
    \item \texttt{Idle}
    \item \texttt{Washing}
    \item \texttt{Rinsing}
    \item \texttt{Spinning}
    \item \texttt{Done}
  \end{itemize}
\end{frame}

\begin{frame}{Diagrama de Transiciones (Conceptual)}
  Secuencia de estados y transiciones:
  \begin{center}
  (Inicio) $\rightarrow$ \texttt{Idle} $\xrightarrow{\text{inicio/tiempo}}$ \texttt{Washing} $\xrightarrow{\text{tiempo}}$ \texttt{Rinsing} $\xrightarrow{\text{tiempo}}$ \texttt{Spinning} $\xrightarrow{\text{tiempo}}$ \texttt{Done} $\xrightarrow{\text{tiempo}}$ \texttt{Idle}
  \end{center}
\end{frame}

%===============================================================================
% Sección 3: Construyendo el Modelo en Simulink y Stateflow
%===============================================================================
\section{Construcción del Modelo}

\begin{frame}{Proceso Paso a Paso en Simulink/Stateflow}
  \begin{enumerate}
    \item Abre Simulink, crea un modelo nuevo.
    \item Añade un bloque \texttt{Chart} (librería Stateflow).
    \item Haz doble clic en el bloque \texttt{Chart} para abrir el editor de Stateflow.
    \item \textbf{Añadir Estados:} Crea los 5 estados: \texttt{Idle}, \texttt{Washing}, \texttt{Rinsing}, \texttt{Spinning}, \texttt{Done}.
    \item \textbf{Transición por Defecto:} Añade una transición por defecto a \texttt{Idle}.
    \item \textbf{Añadir Transiciones Secuenciales:} Dibuja las flechas entre los estados en orden.
  \end{enumerate}
\end{frame}

\begin{frame}{Definiendo Entradas, Salidas y Datos}
  \begin{itemize}
    \item En el editor de Stateflow, define:
    \begin{itemize}
        \item Una **entrada**: \texttt{StartSignal} (boolean/double).
        \item Cinco **salidas**: \texttt{is\_Idle}, \texttt{is\_Washing}, \texttt{is\_Rinsing}, \texttt{is\_Spinning}, \texttt{is\_Done} (boolean/double).
        \item Tres **constantes locales**: \texttt{T\_Washing = 20;}, \texttt{T\_Rinsing = 15;}, \texttt{T\_Spinning = 30;} (double).
    \end{itemize}
  \end{itemize}
\end{frame}

\begin{frame}{Configurando Transiciones}
  Haz doble clic en cada flecha y escribe la etiqueta:
  \begin{itemize}
    \item \texttt{Idle} $\rightarrow$ \texttt{Washing}: \\
    \texttt{[StartSignal == 1] after(2, sec)}
    \item \texttt{Washing} $\rightarrow$ \texttt{Rinsing}: \\
    \texttt{after(T\_Washing, sec)}
    \item \texttt{Rinsing} $\rightarrow$ \texttt{Spinning}: \\
    \texttt{after(T\_Rinsing, sec)}
    \item \texttt{Spinning} $\rightarrow$ \texttt{Done}: \\
    \texttt{after(T\_Spinning, sec)}
    \item \texttt{Done} $\rightarrow$ \texttt{Idle}: \\
    \texttt{after(5, sec)}
  \end{itemize}
\end{frame}

\begin{frame}{Añadiendo Acciones `entry`}
  Haz doble clic en cada estado y añade la acción \texttt{entry:}.
  \begin{itemize}
    \item **Estado Idle:** \\ \texttt{entry: is\_Idle = 1; is\_Washing = 0; is\_Rinsing = 0; is\_Spinning = 0; is\_Done = 0;}
    \item **Estado Washing:** \\ \texttt{entry: is\_Idle = 0; is\_Washing = 1; is\_Rinsing = 0; is\_Spinning = 0; is\_Done = 0;}
    \item **Estado Rinsing:** \\ \texttt{entry: is\_Idle = 0; is\_Washing = 0; is\_Rinsing = 1; is\_Spinning = 0; is\_Done = 0;}
    \item **Estado Spinning:** \\ \texttt{entry: is\_Idle = 0; is\_Washing = 0; is\_Rinsing = 0; is\_Spinning = 1; is\_Done = 0;}
    \item **Estado Done:** \\ \texttt{entry: is\_Idle = 0; is\_Washing = 0; is\_Rinsing = 0; is\_Spinning = 0; is\_Done = 1;}
  \end{itemize}
\end{frame}

%===============================================================================
% Sección 4: Simulación y Visualización
%===============================================================================
\section{Simulación y Visualización}

\begin{frame}{Conectando y Simulando en Simulink}
  \begin{enumerate}
    \item Cierra el editor de Stateflow.
    \item Añade un bloque \texttt{Step} y conéctalo a la entrada \texttt{StartSignal}. Configúralo para un cambio de 0 a 1.
    \item Añade un bloque \texttt{Scope} (5 entradas). Conecta las 5 salidas del Stateflow al Scope.
    \item Configura el tiempo de simulación (ej. 100 segundos).
    \item Ejecuta la simulación.
  \end{enumerate}
\end{frame}

\begin{frame}{Visualizando los Resultados en el Scope}
  \begin{itemize}
    \item Abre el Scope. Verás 5 gráficas (una por cada estado).
    \item Cada gráfica mostrará cuándo el estado correspondiente está activo (valor 1).
    \item Observa la secuencia y duración de cada fase del ciclo.
  \end{itemize}
\end{frame}

%===============================================================================
% Sección 5: Conclusiones y Extensiones
%===============================================================================
\section{Conclusiones y Extensiones}

\begin{frame}{Conclusiones}
  \begin{itemize}
    \item Stateflow es útil para modelar lógica secuencial (máquinas de estados).
    \item Permite definir estados, transiciones y acciones.
    \item La lógica temporal (\texttt{after}) es clave para sistemas con temporizaciones.
    \item Integrado en Simulink, facilita la simulación.
    \item El ejemplo de la lavadora muestra un ciclo simple basado en tiempo y una señal de inicio.
  \end{itemize}
\end{frame}

\begin{frame}{Posibles Extensiones del Modelo}
  Se puede añadir complejidad:
  \begin{itemize}
    \item Entradas como Pausa, Cancelar.
    \item Estados adicionales (Llenado, Drenado).
    \item Transiciones basadas en sensores simulados (nivel de agua).
    \item Controlar actuadores simulados (bombas, motor) con acciones \texttt{entry} o \texttt{during}.
  \end{itemize}
\end{frame}

%===============================================================================
% Sección 6: Preguntas
%===============================================================================
\section{Preguntas}



%--- Fin del Documento ---
\end{document}