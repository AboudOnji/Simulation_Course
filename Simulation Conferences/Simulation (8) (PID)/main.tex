\documentclass{beamer}
\usetheme{Berkeley}
\usepackage[utf8]{inputenc}
\usepackage[spanish]{babel}
\usepackage{graphicx}
\usepackage{amsmath}
\usecolortheme{dolphin}

\title{Control de sistemas en Simulink}
\author{Prof. D.Sc. BARSEKH-ONJI Aboud \inst{1}}
\date{\today}
\institute[Institución]{%
  \inst{1}%
  Facultad de Ingeniería \\ % Ajusta el nombre si es necesario
  Universidad Anáhuac México 
}

%--- Inicio del Documento ---
\begin{document}

%--- Diapositiva de Título ---
\begin{frame}
  \titlepage
\end{frame}

%--- Tabla de Contenido (Opcional) ---
 \begin{frame}{Tabla de Contenido}
  \tableofcontents
\end{frame}

%===============================================================================
% Sección 1: Introducción a los Sistemas de Control
%===============================================================================
\section{Introducción}

\begin{frame}{¿Qué es un Sistema de Control?}
  \begin{itemize}
    \item Un sistema de control busca mantener una variable de salida en un valor deseado (punto de ajuste o referencia), a pesar de las perturbaciones.
    \item Ejemplos: Control de temperatura en un horno, control de velocidad en un coche, control de nivel de líquido en un tanque.
    \item Los sistemas de control a menudo usan retroalimentación (feedback) para comparar la salida actual con la referencia y ajustar la entrada al sistema.
  \end{itemize}
\end{frame}

\begin{frame}{Simulink: Una Herramienta para Simulación y Diseño de Control}
  \begin{itemize}
    \item Simulink (parte de MATLAB) es un entorno de diagrama de bloques para modelar, simular y analizar sistemas dinámicos.
    \item Permite visualizar el comportamiento del sistema y diseñar controladores de manera gráfica.
  \end{itemize}
\end{frame}

%===============================================================================
% Sección 2: El Controlador PID - Teoría y Base
%===============================================================================
\section{El Controlador PID}

\begin{frame}{¿Qué es un Controlador PID?}
  \begin{itemize}
    \item PID significa Proporcional-Integral-Derivativo.
    \item Es el tipo de controlador más común en la industria debido a su simplicidad y efectividad para muchas aplicaciones.
    \item Calcula una señal de control ($u(t)$) basada en el error ($e(t)$), que es la diferencia entre la referencia ($r(t)$) y la salida del sistema ($y(t)$): $e(t) = r(t) - y(t)$.
  \end{itemize}
\end{frame}

\begin{frame}{Componentes del Controlador PID}
  \begin{itemize}
    \item **Proporcional (P):** La acción de control es proporcional al error actual.
    $$u_P(t) = K_p \cdot e(t)$$
    $K_p$ es la ganancia proporcional. Una $K_p$ alta reduce el error de estado estacionario y el tiempo de subida, pero puede aumentar el sobreimpulso y la oscilación.

    \item **Integral (I):** La acción de control es proporcional a la integral del error a lo largo del tiempo.
    $$u_I(t) = K_i \int e(\tau) d\tau$$
    $K_i$ es la ganancia integral. La acción integral elimina el error de estado estacionario acumulado, pero puede empeorar la estabilidad y aumentar el sobreimpulso.

    \item **Derivativo (D):** La acción de control es proporcional a la tasa de cambio (derivada) del error.
    $$u_D(t) = K_d \frac{de(t)}{dt}$$
    $K_d$ es la ganancia derivativa. La acción derivativa predice el error futuro y ayuda a amortiguar las oscilaciones y reducir el sobreimpulso. Puede ser sensible al ruido en la señal de error.
  \end{itemize}
\end{frame}

\begin{frame}{Ecuación del Controlador PID}
  La señal de control total $u(t)$ es la suma de las contribuciones Proporcional, Integral y Derivativa:
  $$u(t) = K_p \cdot e(t) + K_i \int e(\tau) d\tau + K_d \frac{de(t)}{dt}$$

  En el dominio de Laplace, la función de transferencia del controlador PID es:
  $$C(s) = K_p + \frac{K_i}{s} + K_d s = K_p \left(1 + \frac{1}{T_i s} + T_d s\right)$$
  Donde $T_i = K_p/K_i$ es el tiempo integral y $T_d = K_d/K_p$ es el tiempo derivativo.

  (Puedes usar la forma estándar o paralela dependiendo de cómo el bloque PID de Simulink implemente las ganancias).
\end{frame}

%===============================================================================
% Sección 3: Modelado del Sistema de Control en Simulink
%===============================================================================
\section{Modelado en Simulink: Nuestro Ejemplo}

\begin{frame}{Diagrama de Bloques del Sistema de Control}
  Aquí vemos un sistema de control típico implementado en Simulink:
  \begin{figure}
    % Asegúrate de que la imagen esté en el mismo directorio o especifica la ruta
    \includegraphics[width=0.9\textwidth]{PID.png}
    \caption{Diagrama de bloques del sistema de control en Simulink.}
  \end{figure}
  Vamos a analizar cada bloque.
\end{frame}

\begin{frame}{Bloques del Modelo en Simulink}
  \begin{itemize}
    \item \textbf{Step:} Genera la señal de referencia $r$. En este caso, es un cambio escalón, simulando una entrada que cambia abruptamente de un valor a otro (ej. encender un interruptor, cambiar la temperatura deseada).
    \item \textbf{Sum:} Calcula el error $e = r - y$. Tiene una entrada positiva para la referencia (+) y una negativa para la retroalimentación (-).
    \item \textbf{Gain:} Multiplica la señal de error (o una señal intermedia) por una constante. En este ejemplo, multiplica por 2.5. Esto podría ser parte del controlador o una ganancia adicional en la cadena directa.
    
  \end{itemize}
\end{frame}

\begin{frame}{Bloques del Modelo en Simulink}
  \begin{itemize}
    
    \item \textbf{PID Controller:} Este es el bloque que implementa la lógica Proporcional, Integral y Derivativa. Toma el error (escalado por el Gain en este caso) como entrada y genera la señal de control $u$.
    \item \textbf{Plant:} Representa el sistema físico que queremos controlar. Está modelado por su función de transferencia en el dominio de Laplace:
    $$G(s) = \frac{1}{s^2+2s+4}$$
    La entrada es la señal de control $u$ y la salida es la variable controlada $y$.
    
  \end{itemize}
\end{frame}

\begin{frame}{Bloques del Modelo en Simulink}
  \begin{itemize}
    
    
    \item \textbf{Feedback Loop:} La conexión que lleva la salida $y$ de vuelta a la entrada negativa del bloque Sum. Cierra el lazo de control.
    \item \textbf{Scope:} Visualiza las señales seleccionadas durante la simulación, típicamente la referencia $r$ y la salida del sistema $y$.
  \end{itemize}
\end{frame}

\begin{frame}{Proceso de Creación del Modelo en Simulink}
\begin{itemize}
    \item Abre MATLAB y escribe \texttt{simulink} para iniciar.
    \item Crea un nuevo modelo en blanco.
    \item Arrastra los bloques necesarios desde el Library Browser a tu modelo (busca por nombre o en las categorías: \texttt{Sources}, \texttt{Math Operations}, \texttt{Continuous}, \texttt{Sinks}).
       \begin{itemize}
           \item Step, Sum, Gain, PID Controller, Transfer Fcn (para la Planta), Scope.
       \end{itemize}
    \item Conecta las entradas y salidas de los bloques como se muestra en el diagrama. Asegúrate de que el bloque Sum tenga la configuración \texttt{+-}.
    \end{itemize}
 \end{frame}

\begin{frame}{Proceso de Creación del Modelo en Simulink}
 \begin{itemize}
      \item Configura los parámetros de cada bloque haciendo doble clic en ellos:
       \begin{itemize}
           \item \textbf{Step:} Define el tiempo y los valores inicial y final.
           \item \textbf{Gain:} Define el valor de la ganancia (2.5).
           \item \textbf{Transfer Fcn:} Ingresa los coeficientes del numerador ([1]) y del denominador ([1 2 4]).
           \item \textbf{PID Controller:} Inicialmente, puedes dejar las ganancias P, I, D en sus valores por defecto (o 0 si es la primera vez que lo usas), ya que las sintonizaremos después. Asegúrate de que el tipo de controlador sea "PID" (o "PI" si solo vas a usar esos términos, aunque el bloque se llama PID Controller).
           \item \textbf{Scope:} Configura el número de entradas si vas a visualizar múltiples señales (ej. 2 para $r$ y $y$).
       \end{itemize}
    \item Configura los parámetros de simulación (Tiempo de parada) en `Modeling` $>$ `Model Settings`.
 \end{itemize}
   
 
\end{frame}

%===============================================================================
% Sección 4: Sintonización del Controlador PID con la App PID Tuner
%===============================================================================
\section{Sintonización del Controlador PID}

\begin{frame}{¿Por qué Sintonizar el PID?}
  \begin{itemize}
    \item Elegir los valores correctos para $K_p, K_i, K_d$ es crucial para el rendimiento del sistema de control.
    \item Un PID mal sintonizado puede llevar a:
    \begin{itemize}
        \item Respuesta lenta.
        \item Sobreimpulso excesivo.
        \item Oscilaciones constantes.
        \item Inestabilidad del sistema.
    \end{itemize}
    \item El proceso de sintonización ajusta las ganancias para lograr un comportamiento deseado (ej. tiempo de subida rápido, bajo sobreimpulso, error de estado estacionario nulo, buena estabilidad).
  \end{itemize}
\end{frame}

\begin{frame}{La App PID Tuner de Simulink}
  \begin{itemize}
    \item Simulink ofrece una herramienta automática e interactiva para sintonizar controladores PID: la **App PID Tuner**.
    \item Esta app analiza el modelo del sistema (la Planta y el lazo de retroalimentación) y propone ganancias iniciales.
    \item Permite ajustar el balance entre rendimiento (rapidez, sobreimpulso) y robustez (estabilidad frente a incertidumbres).
  \end{itemize}
\end{frame}

\begin{frame}{Uso Básico de la App PID Tuner}
\footnotesize
  \begin{enumerate}
    \item Una vez que tu modelo de Simulink con el bloque PID está construido, haz doble clic en el bloque \texttt{PID Controller}.
    \item Haz clic en el botón \texttt{Tune} (o \texttt{Manage and Tune} en versiones más recientes) en la ventana de parámetros del bloque.
    \item Simulink linealizará el sistema (si es necesario) y abrirá la ventana de la app PID Tuner.
    \item La app mostrará una respuesta inicial (ej. respuesta al escalón).
    \item Puedes usar los controles deslizantes (Performance vs. Robustness) para ajustar rápidamente la respuesta, o especificar requisitos de diseño más detallados.
    \item La app te mostrará las ganancias PID resultantes ($K_p, K_i, K_d$).
    \item Haz clic en \texttt{Update Block} (o similar) para aplicar las ganancias sintonizadas a tu bloque PID en el modelo de Simulink.
    \item Cierra la app y ejecuta la simulación en tu modelo para ver la respuesta con las ganancias sintonizadas. Compara con la respuesta inicial.
  \end{enumerate}
  Esto simplifica enormemente el proceso tradicional de sintonización manual o por métodos clásicos.
\end{frame}

%===============================================================================
% Sección 5: Conclusiones
%===============================================================================
\section{Conclusiones}

\begin{frame}{Conclusiones}
  \begin{itemize}
    \item Los controladores PID son fundamentales en ingeniería de control.
    \item Simulink permite modelar sistemas dinámicos y diseñar/simular controladores de forma gráfica.
    \item El bloque PID Controller en Simulink simplifica la implementación de esta lógica de control.
    \item La App PID Tuner es una herramienta poderosa para encontrar eficientemente las ganancias $K_p, K_i, K_d$ adecuadas para tu sistema, mejorando su rendimiento.
    \item La simulación es clave para probar y validar el diseño del controlador antes de implementarlo en un sistema físico.
  \end{itemize}
\end{frame}

%===============================================================================
% Sección 6: Preguntas
%===============================================================================




%--- Fin del Documento ---
\end{document}