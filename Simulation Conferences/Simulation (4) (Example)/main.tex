\documentclass{beamer}
\usepackage{physics}
\usepackage{ragged2e}
\usepackage{times}
\usepackage{array}
\usepackage{amsmath}  % Paquete principal para ecuaciones avanzadas
\usepackage{amssymb}  % Símbolos matemáticos adicionales
\usepackage{amsfonts} % Fuentes matemáticas adicionales
\usepackage{mathtools}
\usepackage{booktabs} % Para formato APA en tablas
\newcolumntype{L}[1]{>{\raggedright\arraybackslash}p{#1}}
%\usecolortheme{dolphin}%crane, beaver,dolphin,wolverine,seagull,fly,albatross
\setbeamertemplate{caption}[numbered]
\usepackage[english]{babel}
\usepackage{caption}
\usepackage{setspace}
\usepackage{changepage}
\usepackage{hyperref}



\captionsetup{
    font=footnotesize, % Tamaño de la letra para las captions
    labelfont=bf,      % Hace "Figura X" en negrita
    labelsep=colon     % Cambia el separador después de "Figura X"
}
\usepackage[style=apa, backend=biber]{biblatex} % Configuración para APA o el estilo deseado
\addbibresource{sample.bib} % Archivo .bib

\justifying

% Tema para la presentación
\usetheme{cambridge} %  "Madrid"  "Berlin", "CambridgeUS","Warsaw" etc.
\setbeamertemplate{caption}[numbered]
\addtolength{\topskip}{-1cm} % Reduce el espacio superior
\usefonttheme{professionalfonts}
% Personalizar el footline
\setbeamertemplate{footline}{%
  \hbox{%
    \begin{beamercolorbox}[wd=0.8\paperwidth,ht=2.25ex,dp=1ex,leftskip=1em]{author in head/foot}%
      \usebeamerfont{author in head/foot}BARSEKH-ONJI Aboud, ORCID: 0009-0004-5440-8092
    \end{beamercolorbox}%
    \begin{beamercolorbox}[wd=0.2\paperwidth,ht=2.25ex,dp=1ex,rightskip=1em]{page number in head/foot}%
      \usebeamerfont{ page number in head/foot}\insertframenumber{} / \inserttotalframenumber
    \end{beamercolorbox}%
  }%
}
% Información del título
\institute{\textbf{Universidad Anáhuac México, Facultad de Ingeniería}}
\title{\textbf {{Example: Export results to Workspace}}}
\author{\textbf{BARSEKH-ONJI Aboud (DSc.)}}
\date \today

\begin{document}

% Página de título
\begin{frame}
    \titlepage
\end{frame}

\begin{frame}{Example 1}
    \justifying
    Use Simulink to solve the following problem for \( 0 \leq t \leq 13 \) 
    
    \begin{equation}
        \dv{y}{t} =10 sin (t)
    \end{equation}
   
    Where: \(y(0)=0\)

Note: use 'Sin(t)' block from sources.  
    
\end{frame}

\begin{frame}{Example 1}
\begin{figure}
    \centering
    \includegraphics[width=0.8\linewidth]{figs/ex1.png}
    \caption{Simulink model for example 1}
    \label{fig:enter-label}
\end{figure}
\end{frame}

\begin{frame}{Example 1}
\begin{figure}
    \centering
    \includegraphics[width=0.8\linewidth]{figs/ex12.png}
    \caption{Simulink model for Example 1 using Clock and To Workspace blocks}
    \label{fig:enter-label}
\end{figure}
    
\end{frame}

\begin{frame}{Example 2}
    Construct a Simulink model to solve

    \begin{equation}
        \dv{y}{t} =-10y+f(t)
    \end{equation}

    Where:  \(y(0)=1\) and \(f(t)=2sin(4t)\) for \( 0 \leq t \leq 3 \)
\end{frame}

\begin{frame}{Example 2}
\begin{figure}
    \centering
    \includegraphics[width=0.8\linewidth]{figs/ex2.png}
    \caption{Simulink model for Example 2}
    \label{fig:enter-label}
\end{figure}
\end{frame}

\begin{frame}{Example 3: Two-Mass Suspension System}
    \justifying
    The following are the eqquations of motion of two-mass suspension model shown in figure (\ref{fig:3})
    
    \begin{equation}
        m_1\Ddot{x_1}=k1(x_2-x_1)+c_1(\Dot{x_2}-\Dot{x_1})
    \end{equation}
    \begin{equation}
        m_2\Ddot{x_2}=-k1(x_2-x_1)-c_1(\Dot{x_2}-\Dot{x_1})+k_2(y-x_2)
    \end{equation}

    Where: \( m_1=250kg, m_2=40kg, k_1=1.5.10^4 N/m, 
    
    k_2=1.5.10^5 N/m, c_1=1917N.s/m \)
\end{frame}

\begin{frame}{Example 3}
\begin{figure}
    \centering
    \includegraphics[width=0.5\linewidth]{figs/ex3.png}
    \caption{Two-Mass suspension model}
    \label{fig:3}
\end{figure}
\end{frame}

\begin{frame}{Example 3}
The equations of motion can be expressed in state-variable form by letting \( z_1=x_1, z2=\Dot{x_1}, z_3=x_2, z_4=\Dot{x_2}\). The equations of motion become: 
\begin{figure}
    \centering
    \includegraphics[width=\linewidth]{figs/eq3.png}
\end{figure}
    
\end{frame}

\begin{frame}{Example 3}

\begin{figure}
    \centering
    \includegraphics[width=0.8\linewidth]{figs/eq5.png}
\end{figure}
    
\end{frame}

\begin{frame}{Example 3}

\begin{figure}
    \centering
    \includegraphics[width=\linewidth]{figs/eq6.png}
\end{figure}
    
\end{frame}
\begin{frame}{Example 3}
\begin{figure}
    \centering
    \includegraphics[width=0.65\linewidth]{figs/eq8.png}
\end{figure}
    
\end{frame}


\end{document}
