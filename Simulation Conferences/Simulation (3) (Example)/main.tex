\documentclass{beamer}
\usepackage{ragged2e}
\usepackage{times}
\usepackage{array}
\usepackage{amsmath}  % Paquete principal para ecuaciones avanzadas
\usepackage{amssymb}  % Símbolos matemáticos adicionales
\usepackage{amsfonts} % Fuentes matemáticas adicionales
\usepackage{mathtools}
\usepackage{booktabs} % Para formato APA en tablas
\newcolumntype{L}[1]{>{\raggedright\arraybackslash}p{#1}}
%\usecolortheme{dolphin}%crane, beaver,dolphin,wolverine,seagull,fly,albatross
\setbeamertemplate{caption}[numbered]
\usepackage[english]{babel}
\usepackage{caption}
\usepackage{setspace}
\usepackage{changepage}
\usepackage{hyperref}


\captionsetup{
    font=footnotesize, % Tamaño de la letra para las captions
    labelfont=bf,      % Hace "Figura X" en negrita
    labelsep=colon     % Cambia el separador después de "Figura X"
}
\usepackage[style=apa, backend=biber]{biblatex} % Configuración para APA o el estilo deseado
\addbibresource{sample.bib} % Archivo .bib

\justifying

% Tema para la presentación
\usetheme{cambridge} %  "Madrid"  "Berlin", "CambridgeUS","Warsaw" etc.
\setbeamertemplate{caption}[numbered]
\addtolength{\topskip}{-1cm} % Reduce el espacio superior
\usefonttheme{professionalfonts}
% Personalizar el footline
\setbeamertemplate{footline}{%
  \hbox{%
    \begin{beamercolorbox}[wd=0.8\paperwidth,ht=2.25ex,dp=1ex,leftskip=1em]{author in head/foot}%
      \usebeamerfont{author in head/foot}BARSEKH-ONJI Aboud, ORCID: 0009-0004-5440-8092
    \end{beamercolorbox}%
    \begin{beamercolorbox}[wd=0.2\paperwidth,ht=2.25ex,dp=1ex,rightskip=1em]{page number in head/foot}%
      \usebeamerfont{ page number in head/foot}\insertframenumber{} / \inserttotalframenumber
    \end{beamercolorbox}%
  }%
}
% Información del título
\institute{\textbf{Universidad Anáhuac México, Facultad de Ingeniería}}
\title{\textbf {{Example: Modeling of train system}}}
\author{\textbf{BARSEKH-ONJI Aboud (DSc.)}}
\date \today

\begin{document}

% Página de título
\begin{frame}
    \titlepage
\end{frame}

% Tabla de contenido
\begin{frame}
    \frametitle{Table of Contents}
    \tableofcontents
\end{frame}

% Inicio de la presentación
\section{Problem}

\subsection{System Characteristics}
\begin{frame}{System Characteristics}
    \justifying
    In this example, we will consider a toy train consisting of an engine and a car, as shown in the figure \ref{fig:1}. Assuming that the train travels only in one direction, we want to apply control to the train so that it has a smooth start-up and stop, and a constant-speed ride. The mass of the engine and the car will be represented by \textbf{M1} and \textbf{M2}, respectively. The two are held together by a spring, which has the stiffness coefficient of \textbf{k}. \textbf{F} represents the force applied by the engine, and the Greek letter \textbf{µ} represents the coefficient of rolling friction.
\end{frame}

\begin{frame}{System Characteristics}
    \justifying
    \begin{figure}
        \centering
        \includegraphics[width=0.8\linewidth]{figs/train.png}
        \caption{Train System}
        \label{fig:1}
    \end{figure}
\end{frame}

\section{Modeling using Free body diagram}
\begin{frame}{Modeling using Free body diagram}
    \justifying
    Real systems are usually quite complex and exact analysis is often impossible. We shall make \textbf{approximations} and reduce the system components to idealized versions whose behaviors are similar to the real components.
\end{frame}

\begin{frame}{Modeling using Free body diagram}
    \begin{block}{}
        \justifying
        From Newton’s law, we know that the sum of the forces acting on a mass equals mass times its acceleration as shown in Figure \ref{fig:2}. In this case, the forces acting on M1 are the spring, the friction, and the force applied by the engine. The forces acting on M2 are the spring and the friction.
        
        In the vertical direction, the gravitational force is canceled by the normal force applied by the ground, so that there will be no acceleration in the vertical direction. Where:
    \end{block}
    \begin{equation}
        B_1=\mu gM_1 \text{  and  } B_2=\mu gM_2
    \end{equation}
\end{frame}

\begin{frame}{Modeling using Free body diagram}
   \begin{figure}
       \centering
       \includegraphics[width=\linewidth]{figs/system2.png}
       \caption{Free body diagram}
       \label{fig:2}
   \end{figure}
\end{frame}

\begin{frame}{Modeling using Free body diagram}
   \begin{figure}
       \centering
       \includegraphics[width=\linewidth]{figs/systemmath.png}
       \caption{Mathematical System}
       \label{fig:3}
   \end{figure}
\end{frame}

\begin{frame}{Modeling using Free body diagram}
   \begin{figure}
       \centering
       \includegraphics[width=0.7\linewidth]{figs/systemmath.png}
       \caption{Mathematical system}
       \label{fig:3}
   \end{figure}
   \begin{equation}
    M_1 \ddot{X}_1 = F - k_1(X_1 - X_2) - gM_1 \mu \dot{X}_1
\end{equation}

\begin{equation}
    M_2 \ddot{X}_2 = -k_2(X_2 - X_1) - gM_2 \mu \dot{X}_2
\end{equation}
\end{frame}

\begin{frame}{Modeling using Free body diagram}
\begin{figure}
       \centering
       \includegraphics[width=0.7\linewidth]{figs/systemmath.png}
       \caption{Mathematical system}
       \label{fig:3}
   \end{figure}
\begin{figure}
    \centering
    \includegraphics[width=\linewidth]{figs/ec.png}
\end{figure}

\end{frame}

\begin{frame}{Remember: State Variables and equations}
    \justifying
    The variable $U(t)$ is controllable at all time instants $t > t_0$. The input $U(t)$ is taken from the input space. The output variable $Z(t)$ is observable at all time instants $t > t_0$, but there is no direct control on the output variables. $Z(t)$ is taken from the universe of output $Z$.

\textbf{State equation}
\[
\dot{X}(t) = A X(t) + B U(t) 
\]

\textbf{Output equation}
\[
Z(t) = C X(t) + D U(t)
\]
\end{frame}
\section{Modeling using state space system}

\begin{frame}{Modeling using state space system}

 \begin{figure}
     \centering
     \includegraphics[width=0.5\linewidth]{figs/ec1.png}
     
 \end{figure}
\end{frame}
\begin{frame}{Modeling using state space system}

\begin{figure}
     \centering
     \includegraphics[width=0.7\linewidth]{figs/ec2.png}
     
 \end{figure}
\end{frame}


\begin{frame}{Modeling using state space system}
\begin{figure}
    \centering
    \includegraphics[width=0.8\linewidth]{figs/State.png}
\end{figure}
\end{frame}


\end{document}
